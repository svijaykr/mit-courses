\documentclass[psamsfonts]{amsart}

%-------Packages---------
\usepackage{amssymb,amsfonts}
\usepackage{enumerate}
\usepackage[margin=1in]{geometry}
\usepackage{amsthm}
\usepackage{theorem}
\usepackage{verbatim}

\bibliographystyle{plain}

\voffset = -10pt
\headheight = 0pt
\topmargin = -20pt
\textheight = 690pt

%--------Meta Data: Fill in your info------
\title{6.857 \\
Network and Computer Security \\
Lecture 14: Pedersen Commitments}

\author{Lecturer: Ronald Rivest\\
Scribe: John Wang}

\begin{document}

\maketitle

\section{Pedersen Commitments}

\subsection{Commitment Scheme Overview}

Commitment Scheme Motivation: auctions. You want to commit to your bid, so you have to have the same bid as before. This can be thought of as a commitment. $Commit(x)$ is a commitment to $x$. $Reveal(x)$ opens the commitment and reveals $x$.

For commitment schemes, you want to have the following properties:
\begin{itemize}
  \item Hiding: $Commit(x)$ shouldn't reveal anything about $x$.
  \item Binding: Can only open commitment in one way, i.e. you can't find $x'$ such that $Commit(x') = Commit(x)$ where $x' \neq x$.
  \item Non-malleability: Can't create $Commit(x+1)$ from $Commit(x)$.
\end{itemize}

\subsection{Pedersen Commitment Scheme}

Our setup is as follows: we need two large primes $p,q$ such that $q | p-1$ (e.g. $p$ is a safe prime so that $p = 2q + 1$). Let $g$ be a generator of order $q$ subgroup of $Z_p^*$. Squares modulo $p = Q_p$ has order $q$. We let $h = g^a$ where $a$ is secret. It's hard to find $a$ because the discrete log problem is hard.

We have $Commit(x)$ where $x \in Z_q$ so $0 \leq x < q$. Sender chooses some random $r \in Z_q$ and we have $Commit(x) = c = g^x h^r \pmod{p}$. To reveal, you give $x,r$ and the receiver checks that $c = g^x h^r \pmod{p}$.

\subsection{Security of Pedersen Commitment Scheme}

Perfectly Hiding: $c = g^x h^r \pmod{p}$. $c$ could be a commitment to any other $x'$ as well. Suppose $g^x h^r \equiv g^{x'} h^{r'} \pmod{p}$. Can this hold? Since $h = g^a$ we have $g^{x + ar} = g^{x' + ar'} \pmod{p}$. Since $g$ is a generator, exponents must be the same modulo the order, so we have $x+ar \equiv x' + ar' \pmod{q}$. This says that $r' \equiv (x - x')/a + r \pmod{q}$.

Note that we can divide modulo $q$ because $q$ is prime, so we can divide modulo $q$, since $a$ is non-zero. We have just argued that $c$ could be a commitment to any value $x$! This seems like maybe I'm at risk of not really being committed to anything.

We will show that we're actually computationally bound, even though it is theoretically possible for the committed value $c$ to be opened in more than one way. In principle, can you open $c$ as $(x,r)$ and $(x', r')$?

Suppose $g^x h^r = g^{x'} h^{r'} \pmod{p}$. Then $x + ar = x' + ar' \pmod{q}$. We get $a = (x - x')/(r' - r) \pmod{q}$ because $r' \neq r$. Computing $a$ is hard because it is the discrete log of $h$ base $g \pmod{p}$. Therefore, you're computationally bound because the sender can't compute $a$. 

\subsection{Malleability}

However, this scheme is malleable. Consider $c = g^x h^r \pmod{p}$. But you can just get a commitment $c' = g c = g^{x+1} h^r \pmod{p}$ which is one higher than the other scheme. If the first person gives his $x,r$ before you do, then you're home free.

\section{Public Key Encryption}

Let $\lambda$ be a security parameter, which approximately gives you the key size. $1^{\lambda} = 1 1 \ldots 1$ is a set of ones of length $\lambda$. For key generation you have the following steps:

\begin{enumerate}
  \item $Keygen(1^{\lambda}) \to (PK, SK)$ which runs in polynomial time $poly(\lambda)$ and it must be randomized.
  \item $Enc(PK, m) \to c$. Takes some message $m \in \mathrm{M} \to c \in \mathrm{C}$ and is probably randomized.
  \item $Dec(SK, c) \to m$. Deterministic operation which should work for all $(PK, SK)$ pairs and all messages $m$. 
\end{enumerate}

\subsection{ElGamal Public Key Encryption}

Let $G = \langle g \rangle$ be a cyclic group with generator $g$. $G \in Z_p^*$ where $|p| = \lambda$ bits. 

Keygeneration: We pick $x$ at random from $[0, |G| - 1]$ and let $SK = x$, $PK = g^x \pmod{p}$. We then output $(PK, SK)$.

Encryption: Take $m \in G$ and pick $k$ at random from $[0, |G| -1]$. Recipient's PK is $y = g^x \pmod{p}$. The shared secret key is going to be $y^k = g^{x}^k = g^{xk} \pmod{p}$. Output $(g^k, m y^k)$.

Decryption: We have $c = (a,b)$ be the received ciphertext. We can compute $y^k = g^{xk} = (g^k)^x$ because we get $g^k = a$. Therefore, we can get our message back by doing $m = b/a^x$.

Like DH key exchange, the receiver provides $y = g^{x}$ which is PK and the sender provides $a=g^k, m g^{xk}$.

\subsection{Security of ElGamal}

Let's analyze whether this is a secure scheme. Here are the phases of the game:

\begin{itemize}
  \item Phase I (``Find''): Examiner generates $(PK, SK)$ via $Keygen(1^{\lambda})$. $PK$ is then sent to the adversary. Adversary computes (in time polynomial in $\lambda$) and outputs two messages $m_0, m_1$ of the same length and state information.
  \item Phase II (``Guess''): Examiner picks $b \leftarrow \{0,1\}$ which is randomly generated and computes $c_{*} = Enc(PK, m_b)$. The result $c_*$ is given to the adversary. Adversaries compute for time (polynomial in $\lambda$) and outputs his guess $\hat{b}$ for $b$. The adversary wins if $\hat{b} = b$.
\end{itemize}

In this game it takes no skill to win with probability $\frac{1}{2}$. The scheme is \emph{semantically secure} if the probability that the adversary wins is less than $\frac{1}{2} + \epsilon$ where $\epsilon = 2^{-\lambda}$ which is the probability of randomly choosing $SK$ correctly.

Recall computational Diffie-Hellman (CDH) which is computing $g^{ab}$ from $g^a$ and $g^b$. We now have Decision Diffie-Hellman (DDH): distinguishing $(g^a, g^b, g^{ab})$ from $(g^a, g^b, g^c)$ is hard. These problems are related and $DDH$ implies $CDH$. 

Theorem: ElGamal is semantically secure if and only if DDH holds in $G$.

\end{document}
