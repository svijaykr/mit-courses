\documentclass[psamsfonts]{amsart}

%-------Packages---------
\usepackage{amssymb,amsfonts}
\usepackage[all,arc]{xy}
\usepackage{enumerate}
\usepackage[margin=1in]{geometry}
\usepackage{amsthm}
\usepackage{theorem}
\usepackage{verbatim}
\usepackage{tikz}
\usetikzlibrary{shapes,arrows}

\newenvironment{sol}{{\bfseries Solution:}}{\qedsymbol}
\newenvironment{prob}{{\bfseries Problem:}}

\bibliographystyle{plain}

\voffset = -10pt
\headheight = 0pt
\topmargin = -20pt
\textheight = 690pt

%--------Meta Data: Fill in your info------
\title{6.033 \\
Computer Systems Engineering \\
Recitation 10: Trusting Trust}

\author{John Wang}

\begin{document}

\maketitle

Interpreter: Source code gets sent to the interpreter, and the input is sent to the interpreter as well. The output comes out of the interpreter.

Compiler: Source code goes to the compiler, which produces a binary. The input goes to the binary, and this combination produces an output.

\section{Adding Features in C}

Let's say we add a feature to the C source. Originally we have had some binary file $C_0$ when its compiled. Now, when we add a new feature in the source code, we can create a new compiled binary file $C_1$. Thus, both source files can keep working on $C_1$ if we've made a backwards compatible extension.

We can keep using old C programs with the new compiled binary, and we should see the same results as before. At this point, we could consider deleting the old C source, because we now have the new source.

Now let's say we use the new feature in a new extension. We need to compile it with $C_1$, which will give us a new compiled version $C_2$.

Note: the compiler should be able to compile its own source code and produce the same compiled binary (compiler).

\subsection{Thompson's New Feature}

\begin{verbatim}
c = next();
if (c != '\\')
  return c;
c = next();
if (c == 'n')
  return '\n';
if (c == 'v')
  return '\v';
\end{verbatim}

\subsection{Thompson's Trojan Horse}

First, there is the regular C compiler $C_0.c$ and that compiles to $C_0$ binary. There is also $login.c$ which gets compiled by $C_0$ to $login$. Now we create a new program $C_1.c$ which overwrites the original login file. Compiling this new program gives $C_1$.

Now, let's say $C_2.c$ can compile the login program and replace it with a malicious login program. However, it also has a property so that if it is asked to compile $C_1.c$, then it compiles that to the bad compiler.

We can delete the source code of $C_2.c$. Whenever the compiler is recompiled, the special cases will be reinserted.

\begin{verbatim}
def compile(s):
  if s == "gcc.c":
    compile("C2.c")
    return
  if s == "login.c":
    compile("bad_login.c")

  ... (whatever a compiler normally does)
\end{verbatim}

\section{Preventing Trojan Horses}

Let's say we have two versions of compiler binaries, and we have a compiler source. Let's feed the compiler source to produce new compilers $C_3$ and $C_4$ with each binary. If the binaries are different, it's not neccesarily a bad sign because $C_1$ and $C_2$ binaries are possibly different compiler optimizations.

However, when we feed $C.c$ to $C_3$ and $C_4$, we should get the same binary output from both of them because they are supposed to be the same compiler.

\begin{itemize}
  \item
\end{itemize}

\end{document}
