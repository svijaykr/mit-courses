\answer{4-1. Discreteness is the Better Part of Valor}

\begin{enumerate}[(a)]

  \item We will use an algorithm that uses a hash table in order to take the space and time requirements of the discrete log problem down to $\theta(\sqrt{k})$. This will rely on the assumption that hash tables have expected $\theta(1)$ read and write time (which is a legitimate assumption for reasonable hash functions).

Consider the following algorithm. First, we store $g^i$ for values of $i$ in the range $[0, m = 2^(k/2)]$. Then, for each $j$ in the same range, we check to see if $y (g^{-m})^j$ exists in the table. Formally, we have the following algorithm:

\begin{verbatim}

def discrete_log(p, g, y):
  m = 2^(k/2)
  g_results = {}
  for i in range(0, m):
    g_results[g^i % p] = i

  k = y
  for j in range(0, m):
    if k % p in g_results:
      i = g_results[k % p]
      return (i,j)
    else:
      k = k*inverse(g^m)
\end{verbatim}

Here we see that if any pair $(i,j)$ is such that $g^i (g^{2^(k/2)})^j \equiv g^{i + j (2^{k/2})} \equiv y \pmod{p}$, then we will find it with this algorithm. This is because all possible combinations of $i$ and $j$ are iterated over in this algorithm, with the first loop iterating over all possible values of $i$ and the second loop iterating over all possible values of $j$.

If any particular pair $(i,j)$ could be the correct pair for the discrete log problem with equal probability, then the expected number of pairs we must go through is $2^k$. However, we only need a runtime of $2^{k/2} (1 + 1/2)$ because we expect to finish halfway through the second loop. This is because we must complete the first loop in any given call to the \emph{discrete\_log} function, but each $j$ will have probability $1/m$ of hitting a correct result.

The total space is given by the number of items in the hash table, which is $2^{k/2}$.

  \item To solve, this problem, we noticed that we could solve two smaller but related problems. We noticed that computing $z^s$ or $z^r$ was equivalent to computing $g^{sa}$ and $h^{rb}$ respectively. This is because we have $g^a h^b \equiv z \pmod{p}$ and that $g^a h^b = (g_1^{s})^a (g_1^r)^b = g_1^{sa + rb}$. Now, this implies for $z^s$ (and equivalently for $z^r$ by symmetry):

\begin{eqnarray}
  z^s &\equiv& (g_1^{sa+rb})^s \pmod{p} \\
      &\equiv& g_0^{2(sa + rb)s} \pmod{p} \\
      &\equiv& g_0^{2sa} (g_0^{2rs})^b \pmod{p} \\
      &\equiv& (g_0^{2})^{sa} 1^b \pmod{p} \\
      &\equiv& (g_1^{s})^a \pmod{p} \\
      &\equiv& g^{sa} \pmod{p}
\end{eqnarray}

Thus, we can take discrete logarithms for $x$ in the congruence $g^{x} \equiv z^s \pmod{p}$, since we know $x$ only takes on a total of $r$ values. Because of this, we only need to use an algorithm which iterates over the group with $r$ elements, which is relatively fast. Then, once we have found $x$, we can obtain $a$ by taking $a \equiv x s^{-1} \equiv s a s^{-1} \pmod{r}$. A symmetric argument can be applied to $b$.

To solve the congruence $g^x \equiv z^s \pmod{p}$, we use a fast discrete log algorithm called baby-step giant-step. The algorithm iterates through all values of $g^{i}$ for $i \in [0, \sqrt{r}]$, and stores them in a hash table. Then, the algorithm checks if any values of $z^s g^{-j \sqrt{r}}$ are in the hash table for $j \in [0, \sqrt{r}]$. This works because you can decompose $x = jr + i$. This algorithm requires $O(\sqrt{r})$ time and space complexity when computing $g^x \equiv z^s \pmod{p}$ and $O(\sqrt{s})$ time and space complexity when $g^y \equiv z^r \pmod{p}$. Since $r,s \approx \sqrt{p}$, we can see that the total time for our algorithm is $O(\sqrt{s} + \sqrt{r}) = O(\sqrt{\sqrt{p}}) = O(p^{1/4})$. Note we could have used any discrete logarithm algorithm which computes relatively quickly (such as Pollard's Rho or Brent's algorithm), but we decided on the baby-step giant-step because of its ease of implementation.

Our group's number is $z = 872037443554961401$ and our results are given below:
\begin{table}[h!]
  \begin{tabular}{c | c c}
    $i$ & $a$ & $b$ \\
    \hline \hline
    40 & 44833 & 308847 \\
    48 & 3972467 & 2996205 \\
    56 & 97799205 & 6351201 \\
    64 & 1789544324 & 1110942352 \\
    72 & 14241181606 & 27418647169
  \end{tabular}
\end{table}

\end{enumerate}
