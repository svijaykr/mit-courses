\documentclass[psamsfonts]{amsart}

%-------Packages---------
\usepackage{amssymb,amsfonts}
\usepackage{enumerate}
\usepackage[margin=1in]{geometry}
\usepackage{amsthm}
\usepackage{theorem}
\usepackage{verbatim}

\bibliographystyle{plain}

\voffset = -10pt
\headheight = 0pt
\topmargin = -20pt
\textheight = 690pt

%--------Meta Data: Fill in your info------
\title{6.857 \\
Network and Computer Security \\
Lecture 12: Finite Group Theory}

\author{Lecturer: Ronald Rivest\\
Scribe: John Wang}

\begin{document}

\maketitle

\section{Group Theory}

\subsection{Orders of Elts}

The most common group we see in cryptography is $Z_{p}^* = \{1,2, \ldots, p-1 \}$. This is a pretty group, very simple, and nice to work with. We say that $order_{n}(a)$ is the order of $a$ modulo $n$, where $a \in Z_{n}^*$. The order is the smallest $t > 0$ such that $a^t \equiv 1 \mod n$.

Fermat's Little Theorem: If $p$ is prime, then $\forall a \in Z_{p}^*$ we have $a^{p-1} \equiv 1 \pmod{p}$.

Euler's Theorem: For all $n$ and forall $a \in Z_{n}^*$, we have $a^{\phi(n)} \equiv 1 \pmod{n}$, where $\phi(n)$ is the cardinality of the group so $\phi(n) = |\{a; 1 \leq a \leq n; gcd(a,n) = 1 \} |$. 

For example: $Z_{10}^* = \{1, 3, 7, 9\}$. This is a multiplicative group, and $\phi(10) = 4$. Just to check, we have $3^{4} \equiv 1 \pmod{10}$.

Notice that the order of $a$ mod $n$ is the length of the periodicity of $a^{i}$ for $i = \{1,2,3,\ldots\}$. We notice that $order_{n}(a) | |Z_{n}^*|$ so that the order always divides the size of the group.

\subsection{Generators}

Notation: define $<a> = \{a^i, i \geq 1 \}$. We have $order_{n}(a) = |<a>|$.

Definition: If $order_{n}(g) = |Z_{n}^*|$, then $g$ is called the generator of $Z_{n}^*$. In other words $<g> = Z_{n}^*$.

Theorem: $Z_{n}^*$ has a generator (i.e. $Z_{n}^*$ is cyclic) if and only $n = 2,4, p^m, 2p^m$ for prime $p$ and $m \geq 1$.

Theorem: If $p$ is prime, then the number of generators mod $p$ is $\phi(p-1)$. For example $p = 11$ then $|Z_{p}^*| = 10$, and we have $\phi(10) = 4$.

Theorem: If $p$ is prime and $g$ is a generator modulo $p$, then $g^{x} \equiv y \pmod{p}$ has a unique solution for $0 \leq x < p -1$ for each $y \in _Z_p^*$. We call $x$ the discrete logairthm of $y$, base $g$ modulo $p$.

The discrete logarithm problem is the problem of computing $x$ from $y$ given some $p$ and $g$. It is assumed that this is hard, no one has found an algorithm which solves this problem quickly.

\subsection{Generate and Test}

Randomly pick $g$ in $Z_{p}^*$ and test its order. If the order is $p-1$, then $g$ is the generator. Fermat's Little Theorem implies that $g^{p-1} = 1 \pmod{p}$ and $g^{d} \not \equiv 1 \pmod{p}$ for $d$ which is a divisor of $p-1$.

Assume we factored $p-1 = q_1^{e_1} q_2^{e_2} \ldots q_k^{e_k}$. Check that $g^{(p-1)/q} \not \equiv 1 \pmod{p}$ for each $q | (p-1)$.

\section{Public Keys}

Idea: Pick a large ``random'' $p$ such that we know the factorization of $p-1$. This is because factoring $p-1$ is hard.

If $p$ and $q$ are both primes and $p = 2q+1$, then $p$ is called a safe prime and $q$ is called a Sophie Germain prime. The factorization of $p-1$ is just $2q+1-1$ so that factorization is $p-1 = 2 \times q$. To get pairing, we find $q$ first which is a prime, then let $p = 2q + 1$ and we test $p$ for primality.

Theorem: If $p$ is a safe prime $p = 2q+1$ ($p-1 = 2-q$) then for all $a \in Z_p^*$ we have $order(a) \in \{1, 2, q, 2q\}$.

To test of $g$ is a generator we know that we check if $g^{p-1} \equiv 1 \pmod{p}$ and $g^2 \not \equiv 1 \pmod{p}$ and $g^q \not \equiv 1 \pmod{p}$, which implies that $order_p(g) = p-1$ and so $g$ is a generator.

If $p = 2q + 1$ and so $p$ is a safe prime then the number of generators mod $p$ is $\phi(p-1) = q-1$.

Find large random prime $q$. Let $p = 2q + 1$ and test $p$ for primality. If not, loop and pick new $q$, otherwise output $p,q,g$. We can find $g$ by picking $g \in Z_{p}^*$ and test for $g$ being a generator. Since there are a lot of generators, we've got high probability of getting a generator.

\subsection{Common Public Key Setup}

We have $p$ prime and $g$ generator of $Z_{p}^*$ which are published. Alice chooses $x$ where $0 \leq x < p-1$ as her secret key. Publishes $g^{x} \equiv y \pmod{p}$ as her public key. Secrecy of $x$ depends on difficulty of computing $x = \log_{g,p} (y)$ (discrete logarithm problem).

\end{document}
