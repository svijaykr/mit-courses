\documentclass[12pt]{article}

\usepackage[margin=1.25in]{geometry}

\title{Cascading Tree Sheets:\\ 
A Templating Language for the Web\\
6.UAP Final Report
}

\author{
John J. Wang\\
Department of Electrical Engineering and Computer Science\\
wangjohn@mit.edu
}
\date{\today}

\begin{document}
\maketitle

\section{Introduction}

Modern web development can be quite challenging. A web developer needs to be a jack of all trades, understanding everything from system adminstration to design. In fact, a web developer is typically expected to create, deploy, then maintain a website. This requires a diverse set of skills. Spinning up servers and writing CSS could conceivably both be part of a web developer's job description.

A web developer requires a broad range of skills, and one key challenge of this is that understanding and adapting content to a website is a difficult and cumbersome task on top of everything else a developer faces. Web developers have become the de-facto resource for publishing content, as well as creating an environment in which content can be seen.

In earlier eras, there was always separation between the people creating content and the people creating an environment for the content to live in. For example, book printers took the words that were given to them and transcribed them into a new medium. Other examples of this separation come from newspapers and magazines. Editors designed the layout and made small, superficial changes to the content, while writers actually produced the content.

This separation of concerns made it easier for each type of worker to focus on their specialty. The book printer did not need to study the intricacies of language, he or she only needed to transcribe the words. This separation helped make content publishing more efficient.

However, the internet has unhinged the barrier between content producer and layout creator, shifting extra jobs to the web developer. This extra load has made it difficult to produce websites with both high quality content and a high quality layout. Hypertext Markup Language (HTML) was created to allow a user to structure content on the internet. However, the ability to actually structure content on the web was not fully decoupled from the layout of a webpage. This prevented the full separation between content producers and layout creators.

This fundamental problem in the internet spans affects more than just web developers, however. It prevents pure content creators from being able to publish content to the internet easily and without guidance. Content creators now also must learn the basics of web development because their content's HTML structure will provide a basis for a website's layout.

This paper will describe Cascading Tree Sheets (CTS), a language designed to completely separate content from layout on the internet.

\section{CTS Language}

\end{document}
