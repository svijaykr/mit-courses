\documentclass[psamsfonts]{amsart}

%-------Packages---------
\usepackage{amssymb,amsfonts}
\usepackage{enumerate}
\usepackage[margin=1in]{geometry}
\usepackage{amsthm}
\usepackage{theorem}
\usepackage{verbatim}
\usetikzlibrary{shapes,arrows}

\bibliographystyle{plain}

\voffset = -10pt
\headheight = 0pt
\topmargin = -20pt
\textheight = 690pt

%--------Meta Data: Fill in your info------
\title{6.033 \\
Computer Systems Engineering \\
Lecture 14: Fault-Tolerance}

\author{John Wang}

\begin{document}

\maketitle

\section{Strong Form of Modularity: Client/Server}

Limits propagation of effects, used in a single computer using OS or in a network using the Internet. The limitations is that it only isolates benign mistakes like programming errors and there is no recovery plan.

Unfortunately, client/server doesn't provide fault tolerance. We want to be able to extend Client/Server to handle failure. Can we do better than just returning an error, for example, keep computing despite failures? Can we defend against malicious failures (attacks) as well?

\section{Fault Tolerant Computing}

\subsection{Threats}

\begin{itemize}
  \item Software faults (programming mistakes).
  \item Hardware faults (e.g. disk overheats and fails).
  \item Design faults.
  \item Operational faults (most common fault, e.g. fat fingers or tripping over cables).
  \item Environmental faults (earthquakes or tsunamis).
\end{itemize}

\end{document}
