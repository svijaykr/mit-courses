\documentclass[12pt]{article}

\usepackage[margin=1.25in]{geometry}
\usepackage{listings}

\title{EagerDB:\\
A Predictive Cache Warming Tool\\
6.885 Final Paper
}

\author{
John J. Wang\\
wangjohn@mit.edu\\
}
\date{\today}

\begin{document}
\maketitle

\abstract{
Many applications increase database performance using some form of caching. The majority of cache implementations, however, only increase performance for queries that have been already seen. EagerDB takes this a step farther and provides a predictive cache. It predicts queries that will occur in the future with high probability and automatically loads them into the cache. EagerDB allows the user to manually specify query preloads or automatically parse historical database logs, estimate a Markov model, and preload queries which have high transition probabilities.
}

\section{Introduction}

Most web applications improve database performance by caching -- storing data has has previously been used. Using a cache can dramatically improve database performance when temporal locality is present (i.e. when previously requested data is requested multiple times). Caching has become one of the most fundamental concepts in computer science, and in turn, almost all large technology companies use caches in some form.

However, caches assume that historical data repeats. This assumption does not capture all of the potential performance gains that are possible. In fact, there is quite a large amount of data that exists which can be used to improve performance even further. Since there workflows of users tend to be methodical, the queries that get sent to a database can be predicted. One can then estimate the probability of some query $X$ happening given the current state of the database, and ``preload'' $X$ when it has a high probability of happening.

EagerDB provides a framework for preloading.

\end{document}
