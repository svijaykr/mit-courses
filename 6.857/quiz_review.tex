\documentclass[psamsfonts]{amsart}

%-------Packages---------
\usepackage{amssymb,amsfonts}
\usepackage{enumerate}
\usepackage[margin=1in]{geometry}

\bibliographystyle{plain}

\voffset = -10pt
\headheight = 0pt
\topmargin = -20pt
\textheight = 690pt

%--------Meta Data: Fill in your info------
\title{6.857 \\
Network and Computer Security \\
Quiz Review and Notes}

\author{Lecturer: Ronald Rivest\\
Scribe: John Wang}

\begin{document}

\section{Security Scheme Definitions}

Security schemes are normally presented as a game, and cryptographic systems are tested in these games to see if they are secure if an adversary cannot win a disporportionate amount of the time it.

\subsection{IND-CCA (Indistinguishability under Chosen Ciphertext Attack)}

Phase I:
\begin{itemize}
  \item Examiner produces $(PK, SK) \leftarrow Keygen(1^\lambda)$.
  \item Adversary is given $PK$.
  \item Adversary computes in time $poly(\lambda)$ with access to decryption oracle $Dec(SK, \cdot)$ and outputs $m_0, m_1$ where $|m_0| = |m_1|$. The adversary can also store state information $s$ and obtain this information in the next phase.
\end{itemize}

Phase II:
\begin{itemize}
  \item Examiner chooses $b \leftarrow \{0,1\}$ and computes $y = Enc(SK, m_b)$.
  \item Adversary is given access to state information $s$ and allowed to compute in time $poly(\lambda)$. Then, he produces a guess $\hat{b}$.
\end{itemize}

If adversary's advantage, defined as $|P(\hat{b} = b) - \frac{1}{2}|$, is negligible then the encryption scheme is deemed secure.

Note: Encryption must be randomized, and random values cannot be easily observable for IND-CCA security.

\subsection{IND-CCA2 (Indistinguishability under Adaptive Chosen Ciphertext Attack)}

Adapativity is a stronger security claim than IND-CCA. Everything in IND-CCA2 is the same, except that in phase II, the adversary is given access to the decryption block $Dec(SK, \cdot)$ on all inputs except for $y$.

\subsection{IND-CPA (Indistinguishability under Chosen Plaintext Attack)}

\end{document}
