\documentclass[psamsfonts]{amsart}

%-------Packages---------
\usepackage{amssymb,amsfonts}
\usepackage[all,arc]{xy}
\usepackage{enumerate}
\usepackage{mathrsfs}
\usepackage[margin=1in]{geometry}
\usepackage{thmtools}
\usepackage{verbatim}
\usepackage{multirow}


%--------Theorem Environments--------
%theoremstyle{plain} --- default
\newtheorem{prob}{Problem}[section]
\newtheorem{thm}{Theorem}[section]
\newtheorem{cor}[thm]{Corollary}
\newtheorem{prop}[thm]{Proposition}
\newtheorem{lem}[thm]{Lemma}
\newtheorem{conj}[thm]{Conjecture}
\newtheorem{quest}[thm]{Question}

\newenvironment{sol}{{\bfseries Solution}}{\qedsymbol}


\theoremstyle{definition}
\newtheorem{defn}[thm]{Definition}
\newtheorem{defns}[thm]{Definitions}
\newtheorem{con}[thm]{Construction}
\newtheorem{exmp}[thm]{Example}
\newtheorem{exmps}[thm]{Examples}
\newtheorem{notn}[thm]{Notation}
\newtheorem{notns}[thm]{Notations}
\newtheorem{addm}[thm]{Addendum}
\newtheorem{exer}[thm]{Exercise}

\theoremstyle{remark}
\newtheorem{rem}[thm]{Remark}
\newtheorem{rems}[thm]{Remarks}
\newtheorem{warn}[thm]{Warning}
\newtheorem{sch}[thm]{Scholium}

\makeatletter
\let\c@equation\c@thm
\makeatother
\numberwithin{equation}{section}

\bibliographystyle{plain}

\voffset = -10pt
\headheight = 0pt
\topmargin = -20pt
\textheight = 690pt

%--------Meta Data: Fill in your info------
\title{18.781 \\
Problem Set 4.2}

\author{John Wang}

\begin{document}

\maketitle

\section{Problem 1}

\begin{prob}
Find a primitive root modulo 23 and $23^2$. 
\end{prob}

\begin{sol}
Since $23$ is a prime, we know there exists at least one primitive root. We can start computing the orders of $1 < i < 23$. We find:
\begin{eqnarray}
ord_{23}(2) &=& 11 \\
ord_{23}(3) &=& 11 \\
ord_{23}(4) &=& 11 \\
ord_{23}(5) &=& 22 
\end{eqnarray}

Since $\phi(23) = 23 - 1 = 22$, we know that $5$ is a primitive root since $ord_{23}(5) = \phi(23)$. For a primitive root modulo $23^2$, we perform the same search. 
\begin{eqnarray}
ord_{529}(2) &=& 253 \\
ord_{529}(3) &=& 253 \\
ord_{529}(4) &=& 253 \\
ord_{529}(5) &=& 506
\end{eqnarray}

Since $\phi(529) = \phi(23^2) = 23*(23-1) = 506$, we are done, and we have found that $5$ is also a primitive root modulo $23^2$. 
\end{sol}

\begin{prob}
Show that $3^{8} \equiv -1 \pmod{17}$. Explain why this implies $3$ is a primitive root mod 17. 
\end{prob}

\begin{sol}
We know that $3^8 = 9^4 = 81^2$. Moreover, we know that $81^2 = 6561$, and we know that $6561 \equiv 16 \pmod{17} \equiv -1 \pmod{17}$. Now, suppose by contradiction that $3$ is not a primitive root modulo 17. We know from Fermat's Little Theorem that $3^{16} \equiv 1 \pmod{17}$. However, since $3$ is not a primitive root, we know that $3^{k} \equiv 1 \pmod{17}$ where $k < 16$. However, we know that $ord_{17}(3) | 16$ from a lemma proven in class. Since $3^{8} \equiv -1 \pmod{17}$, we know $ord_{17}(3) \neq 8$. Therefore, we must have $ord_{17}(3) = 2,$ or $4$. However, $3^2 = 9 \not \equiv 1 \pmod{17}$ and $3^4 = 81 \not \equiv 1 \pmod{17}$. This is a contradiction. Therefore, $3$ must be a primitive root.
\end{sol}

\section{Problem 2}

\begin{prob}
Let $m$ and $n$ be positive integers with $m$ odd. Show that $(2^m - 1, 2^n+1) = 1$. 
\end{prob}

\begin{sol}
Suppose that $(2^m - 1, 2^n + 1) \neq 1$ by contradiction. Then, there must exist some prime $p$ such that $p | 2^m - 1$ and $p | 2^n + 1$. This means that $2^m \equiv 1 \pmod{p}$ and $2^n \equiv - 1 \pmod{p}$. Moreover, this means that $2^{n} 2^{n} \equiv (-1) (-1) \equiv 1 \pmod{p}$, which means that $2^{2n} \equiv 2^{m} \pmod{p}$. This implies that $m \equiv 2n \pmod{p-1}$ by Euler's theorem. However, since $p-1$ is even, we know that $m$ must also be even, which is a contradiction.
\end{sol}

\section{Problem 3}
\begin{prob}
Show that if $a^k + 1$ is a prime and $a > 1$ then $k$ is a power of $2$. 
\end{prob}

\begin{sol}
First, we will use a lemma.
\begin{lem}
If $k$ is a positive integer, then $(a-b) | (a^k - b^k)$.
\end{lem}

\begin{proof}
First, we can express $a^k - b^k$ as the following:
\begin{eqnarray}
a^k - b^k &=& a^k - b^k + \sum_{i=1}^{k-1} a^i b^{k - i} - \sum_{i=1}^{k-1} a^i b^{k - i} \\
&=& \sum_{i=0}^{k-1} a^{i+1} b^{k-i-1} - \sum_{i=0}^{k-1} a^i b^{k - i} \\
&=& (a - b) \sum_{i=0}^{k-1} a^{i} b^{n - i - 1}
\end{eqnarray}

Therefore, it is clear that $(a-b) | (a^k - b^k)$. 
\end{proof}

Now, we will use the above lemma to prove that $k$ is a power of $2$. First, suppose by contradiction that $k$ is not a power of $2$. Then we can express $k$ as $k = rs$ where $r \in \{1,2, 3, \ldots, k\}$ and $s \in \{1, 3, 5, \ldots, k\}$ where $s$ is odd. Now, we can write $a = 2^r$ and $b = -1$ so that $(a-b) = 2^r + 1$. Define $m = s$, so that $(a^m - b^m) = (2^{r^s} - (-1)^{s}) = 2^{rs} + 1$ since $s$ is odd. We see from the lemma that $(2^r + 1) | (2^{rs} + 1)$ so that $(2^r + 1 ) | (2^k + 1)$ which shows that $a^k + 1$ is not a prime. This is a contradiction, so $k$ must be a power of 2.
\end{sol}

\begin{prob}
Show that if $p|(a^{2^{n}} + 1)$ then $p = 2$ or $p \equiv 1 \pmod{2^{n+1}}$. 
\end{prob}

\begin{sol}
First notice that if $a$ is odd, then $p = 2$ will always work since $a^{2^n} + 1 \equiv 0 \pmod{2}$. Next, if $a$ is even, we see that $a^{2^n} \equiv -1 \pmod{p}$ so that $(a^{2^n})^{2} \equiv 1 \pmod{p}$. This means that $a^{2^{n+1}} \equiv 1 \pmod{p}$. We want to show that $2^{n+1} = ord_p(a)$, since if this is the case, then $2^{n+1} | \phi(p)$ by a theorem shown in class, which implies $2^{n+1} | p - 1$ which can be rewritten as $p \equiv 1 \pmod{2^{n+1}}$. 

So suppose by contradiction that $2^{n+1} \neq ord_p(a)$. Then there must exist some $k$ such that $a^k \equiv 1 \pmod{p}$ where $k < 2^{n+1}$ and $k | 2^{n+1}$. If this is the case, then $k$ clearly must be a multiple of 2, since it divides $2^{n+1}$. This means that $k = 2^{m}$ for some $m < n+1$. This means we must have $a^{2^m} \equiv 1 \pmod{p}$, where $m \leq n$. However we know that $(a^{2^m})^{(2^i)} \equiv 1 \pmod{p}$ for all $i \geq 0$. Rewritten, we see that:
\begin{eqnarray}
a^{2^{m + i}} \equiv 1 \pmod{p} \hspace{0.5cm} \text{where } m \leq n \hspace{0.3cm} \forall i \geq 0
\end{eqnarray}

However, this implies that $a^{2^n} \equiv 1 \pmod{p}$ since $n \geq m$ and $m + i = n$ for some $i \geq 0$. Clearly this is not true, so we have reached a contradiction. From here $p \equiv 1 \pmod{2^{n+1}}$ follows from the argument above.
\end{sol}

\section{Problem 4}

\begin{prob}
Let $a$ and $n > 1$ be any integers such that $a^{n-1} \equiv 1 \pmod{n}$, but $a^{d} \not \equiv 1 \pmod{n}$ for every proper divisor $d$ of $n - 1$. Prove that $n$ is prime.
\end{prob}

\begin{sol}
First, we must have $(a,n) = 1$, otherwise there would be a prime such that $p|a$ and $p|n$ which would imply $a^{n-1} \equiv 1 \pmod{p}$. Since this is the case, then we can calculate $h = ord_n(a)$. We know that by definition $a^{h} \equiv 1 \pmod{n}$, and by a lemma from class, we know that $h | (n-1)$. Since $h | (n-1)$ and we know that $ h \leq n - 1$, we know that $a^{h} \neq 1 \pmod{n}$ by hypothesis. Therefore, we know that $h = n -1$. However, since $n - 1 = h = ord_n(a)$ and $ord_n(a) | \phi(n)$ by definition, we find that $\phi(n) = n - 1$. This occurs only when $n$ is a prime. This completes the proof.
\end{sol}

\section{Problem 5}

\begin{prob}
Show that the sequence $1^2, 2^2, 3^3, \ldots$ considered modulo $p$ is periodic with smallest period $p(p-1)$. 
\end{prob}

\begin{sol}
Let $j$ be the period. We want to show that $a_{i + j} \equiv a_{i} \pmod{p}$ for all $i$. Using the property of our sequence, namely that $a_i = i^i$, we see that $a_{i+j} = (i + j)^{i + j}$. Therefore, we have:
\begin{eqnarray}
a_{i + j} &=& (i + j)^i (i + j)^j \\
&=& \left( \sum_{x=0}^i {i \choose x} i^x j ^{i - x} \right) \left( \sum_{y=0}^j {j \choose y} i^y j^{j - y} \right) \\
&=& \sum_{x=0}^{i} \sum_{y=0}^j {i \choose x}{j \choose y} i^{x + y} j^{i + j - (x+y)} 
\end{eqnarray} 

Substituting $j = p(p-1)$ into the above expression, and noting that $p(p-1)^{b} \pmod{p} \equiv 0$ for all $b \in \mathrm{N}$, we find:
\begin{eqnarray}
a_{i + p(p-1)} &=& \sum_{x=0}^i \sum_{y=0}^j {i \choose x}{p(p+1) \choose y} i^{x+y} (p(p-1))^{i + p(p-1) - (x+y)} \\
&=& i^{i + p(p-1)} 
\end{eqnarray}
The last expression occurs because the only term without any $p(p-1)$ terms occurs when $x = i$ and $y = p(p-1)$. However, using Fermat's Little Theorem, we know that $i^{p-1} \equiv 1 \pmod{p}$, so that $i^{p(p-1)} = i^{(p-1)^p} \equiv 1^{p} \pmod{p} \equiv 1 \pmod{p}$. Thus, $i^{i + p(p+1)} \equiv i^{i}(1) \equiv i^i \pmod{p}$. Since $a_i = i^i$, we have shown that $a_{i + p(p-1)} \equiv a_i \pmod{p}$ which is sufficient to show that the sequence $1^1, 2^2, 3^3, \ldots$ is periodic with period $p(p-1)$. 
\end{sol}

\section{Problem 6}

\begin{prob}
Suppose $(10a, q) = 1$, and that $k$ is the order of 10 $\pmod{q}$. Show that the decimal expansion of the rational number $a/q$ is periodic with smallest period $k$. 
\end{prob}

\begin{sol}
Note that computing $a/q$ yields the following recurrence, where $r_i$ is the $i$th remainder when divided by $q$:
\begin{eqnarray}
10 r_n &=& qk_{n+1}  + r_{n+1} \\
r_0 &=& a
\end{eqnarray}

This can be rewritten in modulo form as $10 r_n \equiv r_{n+1} \pmod{q}$. Since $r_0 = a$, we can actually write an explicit formula for this by first substituting $r_0 = a$, then computing $r_{n+1}$ using our recursive formula. This yields:
\begin{eqnarray}
r_{n} \equiv 10^{n} a \pmod{q} 
\end{eqnarray}

Now, since $k$ is the order of 10 modulo $q$, we know that it is the smallest integer such that $10^{k} \equiv 1 \pmod{q}$. Because of this, we know that the remainders are congruent to the following sequence modulo $q$:
\begin{eqnarray}
a, 10a, 10^2a, \ldots, 10^{k-1} a, a, 10a, \ldots, 10^{k-1}a, \ldots
\end{eqnarray}

This follows since $r_n \equiv 10^{n} a \pmod{q}$ and as soon as $n = k$, we have $r_k \equiv a \pmod{q}$ and the sequence begins anew. Clearly, the sequence of remainders has a period of $k$, so the length of the period $\lambda$ of the decimal expansions is at most $k$. To show that the length is exactly $k$, we expand out the rational number $a/q = 0.\overline{q_1 q_2 \ldots q_\lambda}$:
\begin{eqnarray}
\frac{a}{q} &=& \left( \frac{q_1}{10} + \frac{q_2}{10^2} + \ldots + \frac{q_{\lambda}}{10^{\lambda}} \right) + \left( \frac{q_1}{10^{\lambda + 1}} + \frac{q_2}{10^{\lambda + 2}} + \ldots + \frac{q_{\lambda}}{10^{2\lambda}} \right) + \ldots \\
&=& \left( \frac{q_1}{10} + \frac{q_2}{10^2} + \ldots + \frac{q_{\lambda}}{10^{\lambda}} \right) \left( 1 + \frac{1}{10} + \frac{1}{10^2} + \ldots \right) \\
&=& \left( \frac{q_1}{10} + \frac{q_2}{10^2} + \ldots + \frac{q_{\lambda}}{10^{\lambda}} \right)  \left( \sum_{i=0}^\infty \frac{1}{10^{\lambda i}} \right) \\
&=& \left( \frac{q_1}{10} + \frac{q_2}{10^2} + \ldots + \frac{q_{\lambda}}{10^{\lambda}} \right) \left( \frac{10^{\lambda}}{10^{\lambda} - 1} \right) \\
&=& \left( \frac{q_1 10^{\lambda - 1} + q_2 10^{\lambda - 2} + \ldots + q_{\lambda}}{10^{\lambda}} \right) \left( \frac{10^{\lambda}}{10^{\lambda} - 1} \right) \\
&=&  \frac{q_1 10^{\lambda - 1} + q_2 10^{\lambda - 2} + \ldots + q_{\lambda}}{10^{\lambda} - 1}
\end{eqnarray}

Therefore, we have shown that $q | 10^{\lambda} - 1$, which shows that $10^{\lambda} \equiv 1 \pmod{q}$. This means that $k | \lambda$, and since $\lambda \leq k$, we must have $k = \lambda$. 

\end{sol}


\end{document}