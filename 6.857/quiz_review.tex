\documentclass[psamsfonts]{amsart}

%-------Packages---------
\usepackage{amssymb,amsfonts}
\usepackage{enumerate}
\usepackage[margin=1in]{geometry}

\bibliographystyle{plain}

\setcounter{secnumdepth}{3}
\voffset = -10pt
\headheight = 0pt
\topmargin = -20pt
\textheight = 690pt

%--------Meta Data: Fill in your info------
\title{6.857 \\
Network and Computer Security \\
Quiz Review and Notes}

\author{Lecturer: Ronald Rivest\\
Scribe: John Wang}

\begin{document}

\maketitle

\section{Security Scheme Definitions}

Security schemes are normally presented as a game, and cryptographic systems are tested in these games to see if they are secure if an adversary cannot win a disporportionate amount of the time it.

\subsection{IND-CCA (Indistinguishability under Chosen Ciphertext Attack)}

\emph{Phase I:}
\begin{itemize}
  \item Examiner produces $(PK, SK) \leftarrow Keygen(1^\lambda)$.
  \item Adversary is given $PK$.
  \item Adversary computes in time $poly(\lambda)$ with access to decryption oracle $Dec(SK, \cdot)$ and encryption oracle $Enc(PK, \cdot)$. Adversary outputs $m_0, m_1$ where $|m_0| = |m_1|$. The adversary can also store state information $s$ and obtain this information in the next phase.
\end{itemize}

\emph{Phase II:}
\begin{itemize}
  \item Examiner chooses $b \leftarrow \{0,1\}$ and computes $y = Enc(SK, m_b)$.
  \item Adversary is given access to state information $s$ and allowed to compute in time $poly(\lambda)$. Then, he produces a guess $\hat{b}$.
\end{itemize}

If adversary's advantage, defined as $|P(\hat{b} = b) - \frac{1}{2}|$, is negligible then the encryption scheme is deemed secure.

Note: Encryption must be randomized, and random values cannot be easily observable for IND-CCA security.

\subsection{IND-CCA2 (Indistinguishability under Adaptive Chosen Ciphertext Attack)}

Adapativity is a stronger security claim than IND-CCA. Everything in IND-CCA2 is the same, except that in phase II, the adversary is given access to the decryption block $Dec(SK, \cdot)$ on all inputs except for $y$.

\subsection{IND-CPA (Indistinguishability under Chosen Plaintext Attack)}

Almost the same as IND-CCA, but it is a little weaker. Under IND-CPA, the adversary is given access only to the encryption block, and never to the decryption block. Thus, the adversary can compute any encryptions in $poly(\lambda)$, but cannot use $Dec(SK, \cdot)$ for either Phase I or II.

Again, this security scheme requires that encryption is randomized.

\subsection{Semantic Security}

Semantic Security is equivalent to IND-CPA.

This means that if you want to show IND-CPA, then it is sufficient to show semantic security, and if you want to show semantic security, it is sufficient to show IND-CPA.

\emph{Definition:} A cryptosystem is \emph{semantically secure} if any probabilistic polynomial time algorithm that is given the cipher $c$ of message $m$ and given $|m|$, cannot determine any other information about the message with non-neglible probability. In other words, it must be infeasible for a computationally bounded adversary to obtain significant information about a plaintext from a ciphertext and the public encryption key.

Note that semantic security does not consider CCA where the attacker is able to request the decryption of chosen ciphertexts.

Examples of semantically secure algorithms:
\begin{itemize}
  \item Goldwasser Micali
  \item El Gamal
  \item Paillier
\end{itemize}

\newpage

\section{Commitment Schemes}

Commitment schemes are schemes in which people give a bid and commit to that bid beforehand, yet other people are unable to see that bid until the committer of the bid reveals it. This is especially useful in auctions. You obviously would want some particular properties. These properties are:
\begin{itemize}
  \item Hiding. $Commit(x)$ should not be visible to anyone and one should not be able to gain information about $x$ from $Commit(x)$.
  \item Non-malleability. Can't produce a related commitment from $Commit(x)$. For example, you shouldn't be able to get $Commit(x+1)$ from $Commit(x)$.
  \item Binding. You can only open $Commit(x)$ and reveal the original $x$ which you committed with.
\end{itemize}

In commitment schemes, we have two functions $Commit(x)$ which creates a commitment to $x$ and $Reveal(c)$ which reveals the commitment $c$.

\subsection{Pedersen Commitment Scheme}

\emph{Setup:} Choose $p,q$ as large primes such that $q | p-1$ and $p$ is a safe prime. For example $p = 2q +1$ would work. Now let $q$ be the generator of the order $q$ subgroup of $Z_p^*$. Let $h = g^a$, where $a$ is secret.

\emph{Commit:} We want some commitment $x$ where $x \in Z_q$. Then we choose $r \in Z_q$ uniformly at random and set $Commit(x) = c = g^x h^r \pmod{p}$.

\emph{Reveal:} The sender reveals $x$ and $r$, and the receiver verifies that $c = g^x h^r \pmod{p}$.

To make sure all the properties are satisfied, let's go through them:

Hiding: Given $c = g^x h^r \pmod{p}$, this doesn't tell us anything about $x$ because any $x'$ could be possible. Just pick some $x' \in Z_q$ and we can find an $r'$ that matches this $x'$ in the following way:
\begin{eqnarray}
  g^{x} h^r &=& g^{x'} h^{r'} \pmod{p} \\
  g^{x} g^{ar} &=& g^{x'} g^{ar'} \pmod{p} \\
    g^{x + ar} &=& g^{x' + ar'} \pmod{p} \\
        x + ar &=& x' + ar' \pmod{q} \\
            r' &=& (x-x')/a + r \pmod{q}
\end{eqnarray}

Binding: Let's see if the sender can reveal the message $c$ in two different ways. He can't, as long as the DLP is hard, because he needs to compute $a$ in order to get $r'$ in the above equation. For $a$, you must solve $h = g^a \pmod{p}$.

Non-malleable: Pedersen doesn't satisfy this. For instance, consider $c = g^x h^r$ and $c' = g c$, then we have $c' = g^{x+1} h^r$ so that $Commit(x+1) = Commit(x)g$. This is clearly mealleable.

\newpage

\section{Cryptographic Systems}

\subsection{ElGamal Encryption}

ElGamal is a public key encryption scheme. The system revolves around a cyclic group $G = \langle g \rangle$ with generator $g$.

\emph{Key Generation}: We pick $x$ uniformly at random from $[0, \ldots, |G|-1]$ and set $SK = x$ and $PK = g^x$.

\emph{Encryption}: Pick $k \leftarrow [0, \ldots, |G|-1]$ uniformly at random and assume $m$ can be represented as an element of $G$. Now let $y = g^x$ be the public key of the recipient, we send $c = (g^k, m y^k)$ as the ciphertext.

\emph{Decryption}: We let $c = (a,b)$ be the received ciphertext. Compute and output $b a^{-x}$ as $m$.

\emph{Proof of Correctness}: We just need to show that $b a^{-x} = m$. But, we know that $b = m y^k$ and $a = g^k$, which implies that $b a^{-x} = m y^k (g^k)^{-x} = m y^k (g^x)^{-k} = m y^k y^{-k} = m$. This proves correctness.

We see that ElGamal encryption is related to the Diffie-Hellman key exchange because you encrypt by multiplying by a DH key and decrypt by dividing by a DH key.

Also we should note some facts about ElGamal encryption. First, ElGamal is \emph{malleable}. Given some encryption $E(m) = (g^k, m y^k)$ we can get a new encryption by taking some constant $c$ and creating $E(m') = (g^k, c m y^k) = (g^k, m' y^k$ where $m' = c m$. More generally, ElGamal is \emph{homomorphic}, i.e. multiplying two ciphertexts will create a new ciphertext which is the multiplication of the original plaintexts. Suppose we're given $c_1 = (g^r, m_1 y^r)$ and $c_2 = (g^s, m_2 y^s)$, then we have $c_1 c_2 = (g^{r+s}, (m_1 m_2) y^{r+s}) = Enc(m_1, m_2)$.

You can re-randomize the encryption by multiplying by $Enc(g^s, y^s)$. For example, if you have some ciphertext $c = (g^r, m y^r)$, you can re-randomize by creating $c' = (g^r g^s, m y^r y^s) = (g^{r+s}, m y^{r+s})$. The new coefficient that you randomly chose is now $r+s$.

Unfortunately, malleability is not the best for some cases. If we want to create something which is IND-CCA2 secure, we need to exclude malleability. The next scheme (Cramer-Shoup) builds off of ElGamal and creates a scheme which is IND-CCA2 secure.

\subsubsection{Cramer Shoup}

Let $G_q$ be a group of prime order $q$. For example, we could use $G_q = Q_p$ where $p = 2q+1$ and $p,q$ are prime.

\emph{Key Generation:}
\begin{itemize}
  \item Pick two number $g_1, g_2 \leftarrow G_q$ uniformly at random from the group.
  \item Pick five numbers $x_1, x_2, y_1, y_2, z \leftarrow Z_q$ uniformly at random from the integer group.
  \item Set $c = g_1^{x_1} g_2^{x_2}$, $d = g_1^{y_1} g_2^{y_2}$, and $h = g_1^z$.
  \item Output $PK = (g_1, g_2, c, d, h)$ and we obtain $SK = (x_1, x_2, y_1, y_2, z)$.
\end{itemize}

\emph{Encryption:}
\begin{itemize}
  \item Select some number $r \leftarrow Z_q$ uniformly at random.
  \item Set $u_1 = g_1^{r}$ and $u_2 = g_2^{r}$.
  \item Compute $e = h^r m$ and $\alpha = H(u_1, u_2, e)$ where $H: G_q^{e} \to Z_q$.
  \item Set $v = c^r d^{r \alpha}$ and return the ciphertext $(u_1, u_2, e, v)$.
\end{itemize}

\emph{Decryption:}
\begin{itemize}
  \item Using $H$, compute $\alpha = H(u_1, u_2, e)$.
  \item Check that $u_1^{x_1 + y_1 \alpha} u_2^{x_2 + y_2 \alpha} = v$. If not equal, reject the message. If equal, return $m = e u_1^{-z}$ as the message.
\end{itemize}

\emph{Proof of Correctness:} Note that $e u_1^{-z} = h^r m u_1^{-z} = h^r m (g_1^{r})^{-z}$. But we have chosen $h$ such that $h = g_1^z$ so that $h^r m (g_1^{r})^{-z} = h^r m (h^{-1})^r = m$. Now to give intuition on the correctness condition, we note that:
\begin{eqnarray}
  (u_1^{x_1} u_2^{x_2})(u_1^{y_1 \alpha} u_2^{y_2 \alpha}) &=& ((g_1^{r})^{x_1}(g_2^{r})^{x_2})((g_1^{r})^{y_1 \alpha} (g_2^{r})^{y_2 \alpha}) \\
                                                           &=& c^r (d^r)^\alpha
\end{eqnarray}

But we have defined $v$ such that $v = c^r d^{r \alpha}$, so we are merely checking that this condition is true.

Notes: Cramer-Shoup satisfies IND-CCA2 security if DDH and if $H$ is target collision resistant. Our strongest notion of security is achievable, but at some cost in terms of speed and conplexity of the algorithm.

\subsection{RSA}

RSA is a public key encryption system.

\emph{Key Generation:} We define $Keygen(1^\lambda)$ as the following process:
\begin{itemize}
  \item Pick two large primes $p,q$ and set $n = pq$.
  \item Pick some number $e \leftarrow Z_{\phi(n)}^*$ and compute $d = e^{-1} \pmod{\phi(n)}$. Note that picking $e$ is equivalent to picking some $e$ from $Z_n^*$ where $gcd(e, \phi(n)) = 1$.
  \item Set the keys as $PK = (n, e)$ and $SK = (d, p, q)$.
\end{itemize}

\emph{Encryption:} We have some message $m \in Z_n$ and we define the encryption operation as $Enc(PK, m) = m^e \pmod{n}$.

\emph{Decryption:} We have some ciphertext $c$ and define decryption as $Dec(SK, c) = c^d \pmod{n}$.

\emph{Proof of Correctness:} Note that by CRT, proving correctness for $(m^e)^d \pmod{n}$ is equivalent to proving correctness for $(m^e)^d \pmod{p}$ and $(m^e)^d \pmod{q}$. WLOG we will prove it correct for $p$. In this case, if $m = 0 \pmod{p}$, then $(m^e)^d \equiv 0 \pmod{p}$ so we have the relation $m = (m^e)^d \pmod{p}$. If $m \neq 0 \pmod{p}$, then we have $(m^e)^d = m^{ed} \pmod{p}$. But we know that $d = e^{-1} \pmod{\phi(n)}$ so that $ed = 1 \pmod{(p-1)(q-1)}$. This implies that $ed = 1 + t(q-1)(p-1) = 1 + u(p-1)$. Therefore, we have $m^{ed} = m^{1+u(p-1)} \pmod{p}$ and by FLT, we have $m^{ed} = m \pmod{p}$. Implies that $(m^e)^d = m \pmod{p}$, which completes the proof.

The basic assumptions underlying RSA is that the RSA problem is hard, i.e that given $c$ and $e$ the computation of $c = m^e \pmod{n}$ is difficult for $n$ large. The best known current way to solve this problem is by factoring $n$ and solving the congruence modulo smaller primes, via the Chinese Remainder Theorem.

\subsubsection{RSA-OAEP}

Regular RSA is not semantically secure (i.e IND-CPA) and its not even randomized. Therefore, it's not IND-CCA2 secure either. In order to get it to IND-CCA2, people have developed a new method for encryption which builds off of RSA. The new method adds some extra bits of padding (in order to check validity) and also some extra random bits. The algorithm is a feistel network which uses random oracles $G$ and $H$ to create a new plaintext message to send into RSA.

Functions have the mapping $G: \{0,1\}^{k_0} \to \{0,1\}^{t + k_1}$ and $H: \{0,1\}^{t + k_1} \to \{0,1\}^{k_0}$, where $t = |m|$ is the length of the message.

Encoding using RSA-OAEP:
\begin{itemize}
  \item Start with $k_0$ number of random bits to add and $k_1$ number of extra bits to add to end of message. We have two starting messages $X_0 = m || 0^{k_1}$ and $Y_0 \leftarrow \{0,1\}^{k_0}$.
  \item Obtain the result $X = X_0 \oplus G(Y_0)$.
  \item Use the result $X$ to compute $Y = H(X) \oplus Y_0$.
  \item The message $m' = X || Y$ is now passed into RSA and encrypted to become $(m')^e \pmod{n}$.
\end{itemize}

Notice that this encoding makes sure to include some random bits $Y$ and does some transformations in the middle so that the bits are scattered about. Decoding is easy:
\begin{itemize}
  \item Take the ciphertext $c' = (m')^e \pmod{n}$ and decrypt regularly in RSA (i.e. $m' = c'^d \pmod{n}$).
  \item Parse the decrypted plaintext for $X$ and $Y$.
  \item Obtain $Y_0$ by doing $Y_0 = H(X) \oplus Y$.
  \item Use the result of $Y_0$ to compute $X_0 = G(Y_0) \oplus X$.
  \item Check to make sure that the last $k_1$ bits in $X$ are all zeros. If this is not true, discard the message as invalid. If it is true, accept the message.
\end{itemize}

RSA-OAEP is IND-CCA2 secure, assuming ROM for $G$ and $H$ and RSA is hard to invert on random inputs.

\section{Diffie-Hellman Key Exchange}

\subsection{Key Exchange Algorithm}

The key idea behind Diffie-Hellman key exchange is that you want to establish a secret key between Alice and Bob, without Eve being able to listen in. If Eve can hear all communication that happens between Alice and Bob, this is somewhat hard, and you need to have a particular scheme that will prevent Eve from obtaining too much information. The Diffie-Hellman algorithm relies on the difficulty of the CDH problem in finite abelian groups.

\emph{Setup:} We have a cyclic group $G = <g>$ with generator $g$. Both $G$ and $g$ are fixed and public. Alice picks some secret $x \leftarrow [0,\ldots,|G|-1]$ are random, and Bob picks some secret $y \leftarrow [0, \ldots, |G|-1]$ at random.

\emph{Exchange:} Alice and Bob set their public keys as $g^x, g^y$ respectively and communicate these to each other.

\emph{Computation of Shared Secret:} To compute the shared secret key, Alice takes Bob's secret key of $g^y$ and exponentiates to get $K = (g^y)^x = g^{xy}$. Likewise, Bob takes Alice's secret key $g^x$ and exponentiates to get $K = (g^x)^y = g^{xy}$ which is a shared secret.

If the adversary doesn't know either $x$ or $y$, then it is hard for him to get $g^{xy}$ based on the difficulty of CDH.

\subsection{Computational Diffie-Hellman (CDH)}

This assumption arises from the Diffie-Hellman key exchange. We assume that if we are given $g^a, g^b$, it is computationaly infeasible to compute $g^{ab}$.

\subsection{Decisional Diffie-Hellman (DDH)}

There is a very related problem to CDH, called DDH. The DDH problem states that given a group $G$ with generator $g$, it is hard/infeasible to decide whether a given triple of elements was generated:
\begin{eqnarray}
  (g^a, g^b, g^c) \textrm{ vs. } (g^a, g^b, g^{ab})
\end{eqnarray}

Where $a,b,c$ are all randomly chosen from the group $G$.

If DDH holds in a group, then you can't even recognize when the group key $g^{ab}$ is given to you.

\subsection{Connection Between DDH and CDH}

\subsubsection{DDH implies CDH}

\emph{Theorem:} If DDH holds, then CDH holds as well.

\emph{Proof:} Given that DDH holds, suppose that CDH does not hold. Then given $g^a, g^b$, it is not computationally intractable to compute $g^{ab}$, and so we can compute $g^{ab}$ in polynomial time. Then, this means that we can distinguish between $g^c$ and $g^{ab}$ if $c \neq ab$ because we can just check if $g^k = g^{ab}$ for some $g^k$. This is in contradiction to DDH. Therefore, we see that if DDH holds, then CDH must hold as well.

\subsubsection{CDH is weaker than DDH}

It is believed that CDH is a weaker assumption than DDH. That is, there are groups for which detecting DDH tuples is easy, but for which solving CDH problems is hard.

\end{document}
