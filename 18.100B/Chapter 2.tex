\documentclass[psamsfonts]{amsart}

%-------Packages---------
\usepackage{amssymb,amsfonts}
\usepackage[all,arc]{xy}
\usepackage{enumerate}
\usepackage{mathrsfs}
\usepackage[margin=1in]{geometry}


%--------Theorem Environments--------
%theoremstyle{plain} --- default
\newtheorem{thm}{Theorem}[section]
\newtheorem{cor}[thm]{Corollary}
\newtheorem{prop}[thm]{Proposition}
\newtheorem{lem}[thm]{Lemma}
\newtheorem{conj}[thm]{Conjecture}
\newtheorem{quest}[thm]{Question}

\theoremstyle{definition}
\newtheorem{defn}[thm]{Definition}
\newtheorem{defns}[thm]{Definitions}
\newtheorem{con}[thm]{Construction}
\newtheorem{exmp}[thm]{Example}
\newtheorem{exmps}[thm]{Examples}
\newtheorem{notn}[thm]{Notation}
\newtheorem{notns}[thm]{Notations}
\newtheorem{addm}[thm]{Addendum}
\newtheorem{exer}[thm]{Exercise}

\theoremstyle{remark}
\newtheorem{rem}[thm]{Remark}
\newtheorem{rems}[thm]{Remarks}
\newtheorem{warn}[thm]{Warning}
\newtheorem{sch}[thm]{Scholium}

\makeatletter
\let\c@equation\c@thm
\makeatother
\numberwithin{equation}{section}

\bibliographystyle{plain}

\voffset = -10pt
\headheight = 0pt
\topmargin = -20pt
\textheight = 690pt

%--------Meta Data: Fill in your info------
\title{Rudin Chapter 2 \\
Solutions}

\author{John Wang}

\begin{document}

\maketitle

\section{Problem 2.11}

\begin{thm}
The distance $d_1(x,y) = (x-y)^2$ is not a metric.
\end{thm}

\begin{proof}
Here, the third requirement for a metric does not hold, namely that $d(x,y) \leq d(x,r) + d(r,y)$. This is because $d_1(x,y) = (x-y)^2 = x^2 - 2xy + y^2$ and $d_1(x,r) + d_1(r,y) = x^2 - 2xr + r^2 +r^2 - 2ry + y^2 = x^2 + y^2 + 2r^2 - 2xr - 2ry$. Thus, one must have $-2xy \leq 2r^2 - 2xr - 2ry$ for all $r \in \mathbb{R}^1$ for $d_1$ to be a metric. This is the same as $xy \geq r(x+y-r)$. However, if one sets $ x = 2$ and $ y = 0$, this inequality does not hold for all values of $r$. For instance, $ 0 \not\geq 1 ( 2 - 1) = 1 $ which shows that $d_1$ is not a metric. 
\end{proof}

\begin{thm}
The distance $d_2(x,y) = \sqrt{|x-y |}$ is a metric.
\end{thm}

\begin{proof}
The first two properties of a metric are easy to prove. We know $d_2(x,y) > 0$ holds for all $x \neq y$ and $d_2(x,x) = 0$ because square roots of positive numbers are always positive. Next, $d_2(x,y) = d_2(y,x)$ because $|x - y| = |y-x|$. Finally, we have $ d_2(x,y) \leq d_2(x,r) + d_2(r,y)$ for all $r \in \mathbb{R}^1$. This is because the triangle inequality for absolute values states that $|x-y| \leq |x-r| + |r-y|$, which means $d_2(x,y) \leq \sqrt{|x-r| + |r-y|} = \sqrt{ d_2(x,r)^2 + d_2(r,y)^2}$. However, by the triangle inequality, we know that $ \sqrt{ d_2(x,r)^2 + d_2(r,y)^2} \leq d_2(x,r) + d_2(r,y)$ because $\sqrt{|x-r| + |r-y|} \leq \sqrt{|x-y|} + \sqrt{|r-y|}$. This means that $d_2(x,y) \leq d_2(x,r) + d_2(r,y)$ for all $r \in \mathbb{R}^1$.  Thus, $d_2$ is a metric. 
\end{proof}

\begin{thm}
The distance $d_3(x,y) = |x^2 - y^2|$ is not a metric.
\end{thm}

\begin{proof}
The first property of metrics does not hold. For instance, if $x = 1$ and $y = -1$, then $x \neq y$, but $d_3(x,y) = 0$, which means $d_3$ is not a metric.
\end{proof}

\begin{thm}
The distance $d_4(x,y) = |x-2y|$ is not a metric.
\end{thm}

\begin{proof}
We know that a metric must have the property $d_4(x,y) > 0$ if $x \neq y$. However, this property does not hold for $x = 2$ and $y = 1$, where $d_4(x,y) = 0$ and $x \neq y$. Thus, $d_4$ is not a metric.
\end{proof}

\begin{thm} 
The distance $d_5(x,y) = \frac{ |x-y|} {1 + |x-y|}$ is a metric. 
\end{thm}

\begin{proof}
We see that $d_5(x,y) = 0 \iff x = y$, and that $d_5(x,y) > 0 $ for all $x,y \in \mathbb{R}^1$. Also, we see that since $|x-y| = |y-x|$, that $d_5(x,y) = d_5(y,x)$. Now, we must prove that $d_5(x,y) \leq d_5(x,r) + d_5(r,y)$. This can be done by looking at the quantity
\begin{equation}
d_5(x,r) + d_5(r,y) - d_5(x,y) = \frac{|x-r|}{1+|x-r|} + \frac{|r-y|}{1+|r-y|} - \frac{|x-y|}{1+|x-y|}
\end{equation}
By expanding out the denominator, one finds that this expression is equal to the following
\begin{eqnarray*}
|x-r|(1+|x-y|)(1+|r-y|) &+& |r-y|(1+|x-y|)(1+|x-r|) \\
&-& |x-y|(1+|x-r|)(1+|r-y|)
\end{eqnarray*}
\begin{equation}
= |x-r||r-y||x-y| + 2|r-y||x-r| + |r-y| + |x-r| - |x-y|
\end{equation}
The first two terms are non-negative because they are products of absolute values. The last term is also non-negative because the triangle inequality states that $0 \leq |x-r| + |r-y| - |x-y|$, which means that the entire expression is non-negative. This shows that the final property of metrics is true, namely that $0 \leq d_5(x,r) + d_5(r,y) - d_5(x,y)$. This shows that $d_5$ is a metric.
\end{proof}


\section{Problem 2.12}

\begin{thm}
If $K \subset \mathbb{R}^1$ consist of $0$ and the numbers $\frac{1}{n}$ for $n = 1,2,3,\ldots$, then $K$ is compact (directly from the definition).
\end{thm}

\begin{proof}
We must show that for every open cover of $K$, there exists a finite subcover. To do this, let $\{G_{\alpha}\}_{\alpha \in A}$ be an open cover of $K$. There must exist an index $\alpha_0$ such that $ 0 \in G_{\alpha_0}$. Since $G_{\alpha_0}$ is open, we know that there exists an $r > 0$ such that $N_r(0) \subset G_{\alpha_0}$. Moreover, by the archimedean principle, we know there exists some $n$ such that $\frac{1}{n} < r$. Thus, we have that $N_{\frac{1}{n}}(0) \subset N_r(0) \subset G_{\alpha_0}$, which means that $N_{\frac{1}{n}}(0) \subset G_{\alpha_0}$. Correspondingly, we know that $\frac{1}{n} \in G_{\alpha_0}$. We can apply the above argument to each member of $K$ such that $n = 1,2,3,\ldots$ and obtain $\frac{1}{n} \in G_{\alpha_n}$. Thus, for all $N >n$, we have $\frac{1}{N} < \frac{1}{n} < r$, meaning that we also have $\frac{1}{N} \in G_{\alpha_0}$. Because of this, for all $ N \leq n$, there exist indices $\alpha_N \in A$ such that $\frac{1}{N} \in G_{\alpha_N}$. Thus, we know that $ K \subset G_{\alpha_0} \cup \cdots \cup G_{\alpha_N}$ for some finite $N$. This shows that every open cover $\{G_{\alpha}\}_{\alpha \in A}$ of $K$ has a finite subcover, and that $K$ is compact.
\end{proof}

\section{Problem 2.13}

\begin{thm}
It is possible to construct a compact set of real numbers whose limit points form a countable set.
\end{thm}

\begin{proof}
Consider the set $E_i$ for $0 \leq i \leq 1$ and  $i \in \mathbb{Q}$ that is defined as $E_i := \{ i + \frac{1}{p}: p \in \mathbb{N} \} \cup \{i \}$. In the case of $i = 0$, $E_i$ is the set consisting of $\frac{1}{n}$ for all $n = 1,2,\ldots$ joined with $0$. Now, we can create a set E that is compact and whose limit points form a countable set by defining for all $i \in \mathbb{Q}$ the following
\begin{equation}
E := {\underset{0 \leq i \leq 1}{\bigcup}} E_i
\end{equation}

This means that the set is closed, because it contains all its limit points. One can see that in this case, all the limit points of $E$ are given by the union of all the limit points of $E_i$. There is only one limit point of $E_i$ by construction, which is $i$. Thus, all the limit points of $E$ are the set of all $0 \leq i \leq 1$ for $i \in \mathbb{Q}$, which is definitely a subset of $E$. Moreover, we can show that the set is bounded. To do this, we must show that there is a real number $M$ and a point $q \in \mathbb{R}$ such that $d(p,q) < M$ for all $p \in E$. This is clearly the case because all elements of $E$ are confined to the closed interval $[0,2]$. By the archimedean principle, there exists an $M \in \mathbb{R}$ such that $2*q < M$ for all $q \in \mathbb{R}$. Since $|p-q| < M$ for all $p \in [0,2]$, we know that $d(p,q) < M$ for all $p \in E$ and $ q \in \mathbb{R}$, as $E \subset [0,2]$. This shows that $E$ is both closed and bounded, and since $E \subset \mathbb{R}$, we know by Heine-Borel that $E$ is compact.

Now, it is easy to show that the limit points of $E$ form a countable set. We have already shown that $E'$ is the set $\{i: 0 \leq i \leq 1, i \in \mathbb{Q}\}$. We also know that the rationals are countable and that $E'$ is an infinite subset of $\mathbb{Q}$. Since every infinite subset of a countable set is countable, we know that $E'$ is countable.  
\end{proof}

\section{Problem 2.14}

\begin{thm}
There is an open cover of the segment $(0,1)$ which has no finite subcover.
\end{thm}

\begin{proof}
All we must do is show that there exists a single open cover without any finite subcovers. To do this, consider the open cover $\{ G_\alpha \}$, where $\alpha \in \mathbb{N}$. Now define each interval as follows $G_\alpha := (\frac{1}{\alpha},1)$. Therefore, we have $\bigcup_{\alpha \in \mathbb{N}} G_\alpha = (0,1)$. Thus, $(0,1) \subset \bigcup_{\alpha \in \mathbb{N}} G_\alpha$, and we have an open cover of the interval $(0,1)$. However, this open cover has no finite subcover. We will show this by contradiction. If there does exist a finite subcover of $\{G_\alpha\}$, then there is some largest number $N \in \mathbb{N}$ such that $(0,1) \subset G_{\alpha_1} \cup \cdots \cup G_{\alpha_N}$. This means that $(0,1) \subset (1,1) \cup \cdots \cup (\frac{1}{N},1)$ and that $(0,1) \subset (\frac{1}{N},1)$ for $N \in \mathbb{N}$. This is a contradiction, which means there does not exist a finite subcover of $\{G_\alpha\}$. Thus, we have provided an example of an open cover of the segment $(0,1)$ which has no finite subcover. 
\end{proof}

\section{Problem 2.16}

\begin{thm}
Regard $\mathbb{Q}$, the set of all rational numbers, as a metric space, with $d(p,q) = |p-q|$. Let $E$ be the set of all $p \in \mathbb{Q}$ such that $2<p^2<3$. Show that $E$ is closed and bounded in $\mathbb{Q}$, but that $E$ is not compact. In addition, $E$ is open in $\mathbb{Q}$. 
\end{thm}
\begin{proof}
To show that $E$ is closed in $\mathbb{Q}$, it is sufficient to show that $E = \mathbb{Q} \cap G$ for some $G \subset \mathbb{R}$ such that $G$ is closed in $\mathbb{R}$. Let $G = \{ x : 2 \leq x^2 \leq 3, x \in \mathbb{R}\}$, then it is easy to see that $E = \mathbb{Q} \cap G$. Now we must show that $G$ is closed in $\mathbb{R}$. To do this, note that $\sqrt{3}$ is an upper bound of $G$ because $p \leq \sqrt{3}$ for every $p \in G$. Thus, $\sqrt{3} = \sup{G}$ because for every $h>0$, $\sqrt{3} - h \in G$. Moreover, we can see that $\sqrt{3} \in G$, which means $\sup{G} \in G$ and that $G$ is closed. Since we have shown that $G$ is closed in $\mathbb{R}$, we now know that $E = \mathbb{Q} \cap G$ is closed in $\mathbb{Q}$. 

To show that $E$ is bounded in $\mathbb{Q}$, we must show that there exists an $M \in \mathbb{R}$ and a $q \in \mathbb{Q}$ such that $d(p,q) < M$ for all $p \in E$. For $p>0$, pick any $q \in \mathbb{Q}$ such that $0 < q < p$, and it is clear that $|p-q| < p$. For $p<0$, pick any $q \in \mathbb{Q}$ such that $p<q<0$, and we have $|p-q| < -p$. Thus, there exists an $M \in \mathbb{R}$ and a $q \in \mathbb{Q}$ such that $d(p,q) < M$ for all $p \in E$, showing that $E$ is bounded in $\mathbb{Q}$.  

We have now shown that $E$ is closed and bounded in $\mathbb{Q}$, but have yet to prove that $E$ is not compact. To do this, we will use the Heine-Borel theorem, and show that $E$ is not closed and bounded in $\mathbb{R}$. To do this, we will show that $E$ is not closed in $\mathbb{R}$, which can be seen if one examines $y = \sup{E}$. One can see that $y = \sqrt{3} = \sup{E}$, because for each $p \in E$, $p \leq \sqrt{3}$, and for every $h>0$, $\sqrt{3} - h \in E$. However, since $\sqrt{3} \notin E$, we can see that $\sup{E} \notin E$, showing that $E$ does not contain all of its limit points. Thus, $E$ is open in $\mathbb{R}$. This implies that $E$ is not compact by Heine-Borel.

Now we want to know whether $E$ is open in $\mathbb{Q}$. To do this, we must know whether every point $p \in E$ is an interior point. In other words, does there exist an $r>0$ such that $N_r(p) \subset E$? If one picks $r = \min \{|-\sqrt{3} - p|, |-\sqrt{2}-p, |\sqrt{2}-p|,|\sqrt{3}-p|\}$, then one will always be able to find a distance $\frac{r}{2}$ such that $N_{\frac{r}{2}}(p) \subset E$ for every $p \in E$. Thus, $E$ is open in $\mathbb{Q}$. 
\end{proof}

\section{Problem 2.22}

\begin{thm}
A metric space is called \textit{separable} if it contains a countable dense subset. We shall show that $\mathbb{R}^k$ is separable. 
\end{thm}

\begin{proof}
First, consider the set $\{ P \}$ of points $p \in \mathbb{R}^k$ such that $ p = (p_1, p_2,\dots,p_k)$ and $p_n \in \mathbb{Q}$ for all $n = 1,\ldots,k$. In other words, $\{P\}$ is the set of points with only rational coordinates, $\mathbb{Q}^k$. We know that $\{P\}$ is countable because $\mathbb{Q}$ is countable, and $\{P\}$ is a finite grouping of rational coordinates. 

Now, we must show that $\{P\}$ is dense in $\mathbb{R}^k$. In other words, each $x \in \mathbb{R}^k$ must have every neighborhood contain a point $q \neq x$ such that $q \in \mathbb{Q}^k$, or $x$ must be an element of $\mathbb{Q}^k$. Given $r>0$, there exists some $q_n \in \mathbb{Q}$ such that $d(x_n,q_n) < \frac{r}{k}$ for $k \in \mathbb{N}$ by the archimedean property. Thus, we know that 
\begin{eqnarray}
d(x,q) &=& \sqrt{d(x_1,q_1)^2 + d(x_2,q_2)^2 + \ldots + d(x_k,q_k)^2} \\
&<& \underbrace{\sqrt{ \left(\frac{r}{k}\right)^2 + \ldots + \left(\frac{r}{k} \right)^2}}_{k}
\end{eqnarray}
Since the expression in (6.3) is equal to $\sqrt{\frac{k r^2}{k^2}} = \frac{r}{\sqrt{k}} \leq r$, where the last inequality comes from the fact that $k \in \mathbb{N}$, we have that $d(x,q) < r$ for each $x \in \mathbb{R}^k$. This shows for all $x \in \mathbb{R}^k$, each neighborhood of $x$ contains a $q \neq x$ such that $ q \in \{ P \}$. This means that $\{P\}$ is dense in $\mathbb{R}^k$, and since we have already shown countability, $\{P\}$ is also separable.  
\end{proof}


\section{Problem 2.17}

\begin{thm}
Let $E$ be the set of all $x \in [0,1]$ whose decimal expansion contains only the digits $4$ and $7$. Then $E$ is not countable.
\end{thm}

\begin{proof}
By contradiction, assume that $E$ is countable. Then one can list all of the elements of $E$ as follows:

\begin{eqnarray*}
&0.s_{11} s_{12} s_{13} \ldots \\
&0.s_{21} s_{22} s_{23} \ldots \\
&0.s_{31} s_{32} s_{33} \ldots \\
&  \vdots
\end{eqnarray*}

Then one can construct a sequence $s = 0.s_1 s_2 s_3 \ldots$ such that $s_k$ is the opposite digit of $s_{kk}$, where $s_{kk}$ is the $k$th digit of the $k$th element of $E$. For example, if $s_{11} = 4$, then $s_1 = 7$. Thus, the sequence $s$ is different than every sequence on the list by at least one digit. Moreover, it is a decimal expansion composed of only the digits $4$ and $7$, so the list does not contain all the elements of $E$. This is a contradiction, so $E$ is not countable.
\end{proof}

\begin{thm}
The set $E$ is not dense in $[0,1]$.
\end{thm}

\begin{proof}
We must show that $\bar{E} \neq [0,1]$. To do this, note that $0.\bar{4} = \inf E$. This is because for any $x \in E$, $0.\bar{4} \leq x$ which means $0.\bar{4}$ is a lower bound and $ 0.\bar{4} \in E$, so $\forall h>0$ one can see that $0.\bar{4} + h$ is not a lower bound of $E$. Thus any number $p < 0.\bar{4}$ is not contained in $E$. Let $P = [0,0.\bar{4})$, then for any $p \in P$, not every neighborhood $N_r(p)$ contains a $q \in E$. This is because there exists an $r = \frac{0.\bar{4} - p}{2}$ such that $N_r(p) \cap E = \emptyset$. Thus, these $p$ are not limit points of $E$ or elements of $E$. Moreover, since $ p \in [0,1]$, we can see that $E$ is not dense. 
\end{proof}

\begin{thm}
The set $E$ is compact.
\end{thm}

\begin{proof}
We will show that $E$ is closed and bounded, and because $E \subset \mathbb{R}$, the Heine-Borel Theorem shows that $E$ is compact. First, it is easy to show that $E$ is bounded because $E \subset [0,1]$. By the archimedean principle, $\exists q,N,r \in \mathbb{R}$ such that $N d(p,q) < r$ for all $p \in [0,1]$ because $|p-q| \in \mathbb{R}$. Thus, $d(p,q) < \frac{r}{N}$ and $\frac{r}{N} \in \mathbb{R}$. Since there always exists a number $\frac{r}{N}$ such that $d(p,q) < \frac{r}{N}$ for $p \in [0,1]$ and $q \in \mathbb{Q}$, we know that $[0,1]$ is bounded. Since $E \subset [0,1]$, we know that $E$ is also bounded.

To show that $E$ is closed, we must show that $E' \subset E$. Let us assume the contrary, namely that there exists an $x \in E'$ such that $x \notin E$. Since $x$ is not in $E$, there is at least one digit that is not a 4 or 7. Let us denote the decimal expansion of $x$ by $ x = x_1 x_2 \ldots x_k \ldots$ where $x_k$ is the digit which is not a 4 or 7. Now we can see that it is not true that $\forall r>0$, $N_r(x)$ contains a $q \neq x$ such that $q \in E$. This is because the neighborhood with radius $r = \left(\frac{1}{10} \right)^{k+1}$ will never contain a $q \in E$. The closest $q \in E$ to the neighborhood with this radius would have a decimal expansion of $ q_{min} = x_1 x_2 \ldots \bar{q_k}$, where $q_k$ is given by $q_k = 7$ if $|7-x_k|< |4-x_k|$ or $q_k = 4$ if $|4-x_k| < |7-x_k|$. Every other $q \in E$ has a further distance from any element $x \in E$, namely $d(x,q_{min}) \leq d(x,q)$. Thus, if $q_{min} \notin N_r(x)$, then no $q \in E$ belongs to $N_r(x)$. 

But it is easy to see that $q_{min} \notin E$ because $ |q_{min} - x| > \left(\frac{1}{10}\right)^{k} ( |q_k - x_k|) > \left( \frac{1}{10} \right)^{k+1}$. This is because $|q_k - x_k| > 1$, as $q_k \in \left\{ 0<p<9: p \in \mathbb{N}, p \neq 4, p \neq 7 \right\} $ and $x_k \in \{4,7\}$. Thus, $E$ is closed. Since we have already shown $E$ is bounded, by the Heine-Borel Theorem, we can conclude that $E$ is compact.
\end{proof}

\begin{thm}
The set $E$ is perfect.
\end{thm}

\begin{proof}
Since we know $E$ is closed by the above theorem, we must show that any $x \in E$ is a limit point of $E$. First, take the decimal expansion $x = 0.x_1 x_2 x_3 \ldots x_n \ldots$ where each $x_1,x_2,x_3, \ldots \in \{4,7\}$. Then let $z_k = 0. q_1 q_2 q_3 \ldots q_n \ldots$ such that

\begin{displaymath}
q_n = \left\{
\begin{array}{ll}
x_k&\text{if } k \neq n \\
4 & \text{if } x_k = 7 \\
7 & \text{if } x_k = 4
\end{array}
\right.
\end{displaymath}
\end{proof}

Thus, $x$ is the same as $z_k$ up to the $k$th digit. After the $k$th digit, $x$ and $z_k$ are completely different. It follows that for every $r>0$, there is a neighborhood $N_r(x)$ that contains a $q \neq x$ such that $q \in E$. This is because one can always find a $z_k$ such that $d(x,z_k) < r$. If the decimal approximation of $r$ begins with a significant digit at the $i$th position, then one can always take $z_{i+1}$, which will be an element of $E$ inside the neighborhood $N_r(x)$. Thus, each point $x$ is a limit point of $E$, making it a perfect set.

\section{Problem 2.18}

\begin{thm}
There is a nonempty perfect set in $\mathbb{R}^1$ which contains no rational number.
\end{thm}

\begin{proof}
Consider the set constructed in the following way. Take the closed interval $[a,b]$ between two irrational numbers $a,b \in \mathbb{I}$. Since the rationals are countable, it is possible to order the rational numbers in the interval into a list. So assume $E = {q_1, q_2, q_3, \ldots, q_n, \ldots}$ is a list of all the rational numbers in the interval $[a,b]$. Now take an open interval $(a_1,b_1)$ where $a_1, b_1 \in \mathbb{I}$ around $q_1$ such that $a < a_1 < q_1 < b_1 < b$. Define the set $A_1 = [a,b] \setminus (a_1,b_1)$. Keep producing these intervals around $q_n$ and define $A_n = A_{n-1} \setminus (a_n, b_n)$. If a point $q_n$ has already been removed, then instead move to the next $A_n$ with the following $A_n = A_{n-1}$. In order to make sure that no endpoints are removed from a different interval being removed, we must have the conditions $q_n - a_n < \max_{i=1}^\infty \{ |q_n - b_i| \}$ and $q_n - b_n < \max_{i=1}^\infty \{ |q_n - a_i | \}$. Now take 
\begin{equation}
A = \bigcap_{n=1}^\infty A_n
\end{equation}
We know that each $A_n$ is closed because the intervals removed were open, and their complements are closed. Moreover, we made sure that no endpoints were accidentally removed by other intervals. Moreover, each $A_n$ is bounded because $A_n \subset [a,b]$ and $[a,b]$ is bounded by the archimedean principle (see Problem 2.17, Theorem 1.3, paragraph 2 for the identical proof which works for any interval in $\mathbb{R}^1$). Thus, we can see that $A_n$ is compact by the Heine-Borel Theorem. Moreover, each $A_n$ is nonempty because there will always exist the irrational endpoints $a,b$. Since $A_n \supset A_{n+1}$ for $n = 1,2,3,\ldots$ and $\{ A_n \}$ is a sequence of nonempty compact sets, then $\bigcap_1^\infty A_n$ is not empty. 

Now we must show that $A$ is a perfect set, and it is sufficient to show that $A$ contains no isolated point. Let $x \in A$ and let $S$ be any segment containing $x$. Let $I_n$ be the interval of $A_n$ which contains $x$. Choose $n$ large enough so that $I_n \subset S$. Then let $x_n$ be an endpoint of $I_n$ such that $x_n \neq x$. It follows from the construction of $A$ that $x_n \in A$. Hence, $x$ is a limit point of $A$ and $A$ is perfect.
\end{proof}

\section{Problem 2.19}

\begin{thm}
If $A$ and $B$ are disjoint closed sets in some metric space $X$, then they are separated.
\end{thm}

\begin{proof}
We must show that $\bar{A} \cap B = \emptyset$ and $A \cap \bar{B} = \emptyset$ if $A \cap B = \emptyset$ with $A$ and $B$ closed. Thus, if $A$ and $B$ are closed, then $A = \bar{A}$ and $B = \bar{B}$. Since $A \cap B = \emptyset$, then we can also see that $\bar{A} \cap B = \emptyset$ and $A \cap \bar{B} = \emptyset$, which shows the two sets are separated.
\end{proof}

\begin{thm}
If $A$ and $B$ are disjoint open sets, then they are separated.
\end{thm}

\begin{proof}
If $A$ and $B$ are disjoint, then $A \cap B = \emptyset$. Thus, we must show that $A' \cap B = \emptyset$ and $A \cap B' = \emptyset$. By contradiction, assume $A' \cap B \neq \emptyset$ without loss of generality. Then there exists some $x \in A'$ which is also an element of $B$. Since $x \in A'$, $x$ is a limit point of $A$ and for every $r>0$, there exists an $a \in N_r(x)$ such that $a \neq x$ and $a \in A$. Since $x \in B$ and $B$ is open, there also exists a neighborhood $N(x) \subset B$. Thus, there must be some $a \in N(x)$ such that $a \neq x$ and $a \in A$. This is a contradiction because then $a$ would be an element of $N(x) \subset B$ and $A$, implying that $A$ and $B$ are not disjoint.
\end{proof}

\begin{thm}
Fix $p \in X$, $\delta > 0$. Define $A$ to be the set of all $q \in X$ for which $d(p,q) < \delta$. Define B similarly, with $>$ in place of $<$. Then $A$ and $B$ are separated.
\end{thm}

\begin{proof}
We must show that $\bar{A} \cap B = \emptyset$ and $A \cap \bar{B} = \emptyset$. Assume the contrary of the first relation, namely that $\bar{A} \cap B \neq \emptyset$. Then there exists an $x \in A \cup A'$ such that $x \in B$. If $x \in A$, then $d(p,q) < \delta$ and $d(p,q) > \delta$ because $x \in B$ as well. This is a contradiction, so $x \notin A$. 

If $x \in A'$, then for all $r>0$, there exists an $s \in N_r(x)$ such that $s \neq x$ and $s \in A$. Thus, $d(x,s) < r$ by the definition of neighborhood and $d(p,s) < \delta$ because $s \in A$. Therefore, by the triangle inequality, we know $d(p,x) \leq d(x,s) + d(p,s) < \delta + r$. Since $x \in B$, we know $d(p,x) > \delta$ so that $ \delta < d(p,x) < \delta + r$ for all $r>0$. As we take smaller and smaller values of $r$, then this expression becomes $\delta < d(p,x) \leq \delta$. This cannot be true because if $d(p,x) = \delta$, then $\delta \nless d(p,x)$ and $d(p,x)$ cannot be greater or less than $\delta$ either, as one part of the inequality would not hold for both cases. This is a contradiction and shows that $\bar{A} \cap B = \emptyset$.   

To show that $A \cap \bar{B} = \emptyset$, we proceed along the same lines and assume the contrary: $A \cap \bar{B} \neq \emptyset$. Then there exists some $x \in B \cup B'$ such that $x \in A$. From the same argument as above, $x \notin B$. Now, if $x \in B'$, then for every $r>0$, there exists an $s \in N_r(x)$ such that $s \neq x$ and $s \in B$. Thus, $d(x,s) < r$ by the definition of neighborhood and $d(p,x) < \delta$ since $x \in A$. By the triangle inequality, we have $d(p,s) < d(x,s) + d(p,x) < \delta + r$. Moreover, since $s \in B$, we know that $d(p,s) < \delta$. This shows that $ \delta < d(p,s) < \delta + r$ for all $r > 0$. By the identical argument as above, this is a contradiction and $A \cap \bar{B} = \emptyset$. Thus, $A$ and $B$ are separated. 
\end{proof}

\begin{thm}
Every connected metric space with at least two points is uncountable.
\end{thm}

\begin{proof}
If a metric space has at least two elements $a,b \in X$, then $d(a,b) > 0$. Choose some $\delta_n = n d(a,b)$ where $n \in (0,1)$. Then let $A_n = \{ p \in X, d(p,a) < \delta_n \}$ and $B_n = \{ p \in X, d(p,a) > \delta_n \}$. Thus, $A_n$ and $B_n$ are nonempty because $0 < \delta_n < d(a,b)$, so $a \in A_n$ and $b \in B_n$ for all $n \in (0,1)$. By the last theorem, we know that $A_n$ and $B_n$ are separated. This means that $A_n \cup B_n \neq X$ because $X$ is connected. Since $A_n \subset X$ and $B_n \subset X$, there must be some $E_n \subset X$ such that $E_n \cap ( A_n \cup B_n) = \emptyset$. The only possible set $E_n$ is allowed to be by construction is $E_n = \{ p \in X, d(p,a) = \delta_n \}$. Now let 
\begin{equation}
E = \bigcap_{n \in (0,1)} E_n.
\end{equation}
Thus it is clear that $E \subset X$. Moreover, $E$ is uncountable because $(0,1)$ is uncountable. Hence, there does not exist a bijective mapping between $E$ and $\mathbb{N}$. Since $E \subset X$ is uncountable, $X$ itself is also uncountable.
\end{proof}

\section{Problem 2.20}

\begin{thm}
The closure of a connected set $E$ is always connected.
\end{thm}

\begin{proof}
Assume the contrary, namely that $\bar{E} = A \cup B$ such that $\bar{A} \cap B = \emptyset$ and $A \cap \bar{B} = \emptyset$. We will show that either $A$ or $B$ must be empty. We can express $E$ as $E = (E \cap A) \cup (E \cap B)$ because $E \subset A \cup B$. Since $A$ and $B$ are separated, we know that $A$ and $B$ are disjoint, so that $(E \cap A)$ and $(E \cap B)$ are disjoint as well. Since $E$ is connected, we must have that $E \cap A = \emptyset$ or $E \cap B = \emptyset$. Assume without loss of generality that $E \cap A = \emptyset$, then $E \subset B$ because $E = E \cap B$. This implies that $\bar{E} \subset \bar{B}$ and further that $\bar{E} = A \cup B \subset \bar{B}$. However, since $A \cap \bar{B} = \emptyset$, we can find that
\begin{equation}
A = A \cap (A \cup B) \subset A \cap \bar{B} = \emptyset
\end{equation} 
which shows that $\bar{E}$ is in fact connected. This completes the proof.
\end{proof}

\begin{thm}
Not all interiors of connected sets are connected. 
\end{thm}

\begin{proof}
Consider the two closed balls $A := \{ (x,y) \in \mathbb{R}^2 : d((x,y),(-1,0)) \leq 1 \}$ and $B := \{ (x,y) \in \mathbb{R}^2: d((x,y), (1,0)) \leq 1 \}$. Then $E = A \cup B$ is a connected set, but its interior will be given by the union of the two open balls $A^\circ \cup B^\circ$. This is not connected because $A \cap B^\circ = \emptyset$ and $A^\circ \cap B = \emptyset$. 
\end{proof}

\section{Problem 2.29}

\begin{thm}
Every open set in $\mathbb{R}^1$ is the union of an at most countable collection of disjoint segments.
\end{thm}

\begin{proof}
We know that $\mathbb{R}^1$ is separable and contains a countable dense subset $E$ by the theorem proven in exercise $2.22$. Then if $x \in \mathbb{R}^1$, $x$ must be either a limit point of $E$ or an element of $E$. Now take an open interval $(a,b) \subset \mathbb{R}^1$. For every $x \in (a,b)$, we must have either $x \in E$ and/or $x \in E'$. Define $A := \{x \in (a,b), x\in E \}$ and $B := \{ x \in (a,b), x \in E', x \notin E \}$, then $A \cup B = (a,b)$. It is clear that $A$ and $B$ are disjoint by their construction. Moreover, they are at most countable because $A \subset E$ and $B \subset E$, and $E$ is countable. If $A$ and $B$ are infinite, then they are an infinite subset of a countable set, which makes them countable. If $A$ and $B$ are finite, then they are at most countable by the definition.

Now it remains to show that we have either 1) $A \neq \emptyset$ and $B \neq \emptyset$ or 2) $B = \emptyset$ and $A$ is a union of disjoint segments (which is still at most countable from the above argument). If we can show that only these two cases occur, then the proof will be completed. Assume by contradiction that $A = \emptyset$, then $E \cap (a,b) = \emptyset$. Thus, we must have $(a,b) = B$ and so $(a,b) \subset E'$. Then for every $x \in (a,b)$ and every $r>0$, there exists a neighborhood $N_r(x)$ containing a $q \neq x$ such that $q  \in E$. Moreover, since $(a,b)$ is open, there exists an $N(x) \subset (a,b)$. Thus, some $q \in N(x)$ is also $q \in (a,b)$. This implies that $E \cap (a,b) \neq \emptyset$ because $q \in E$. This is a contradiction because we assumed $A = \emptyset$ and thus that $E \cap (a,b) = \emptyset$. Thus, we must have $A \neq \emptyset$.

Now we will show that if $B = \emptyset$, then $A$ is a union of disjoint segments. If $B = \emptyset$, then $(a,b) = A$. Moreover, for every $x \in (a,b)$, we have $x \notin E'$ even though $x \in E$, implying that $x$ is an isolated point of $E$. Thus, for some neighborhood $N_\delta(x)$, there does not exist a $q \in N_\delta(x)$ such that $q \neq x$ and $q \in E$. In other words, for some $\delta > 0$, we have $d(x,q) > \delta$ for every $q \neq x$ and $q \in E$. Thus, one can make nonempty partitions $C := \{ q \in A: d(x,q) < \delta \}$ and $D := \{ q \in A: d(x,q) > \delta \}$. By the theorem proven in problem 2.19(c), we know that $C$ and $D$ are separated and hence disjoint. Moreover, $(C \cup D) \cup \{x\} = A$ is a union of disjoint segments. Thus, we have shown that if $B$ is empty, the open set $(a,b)$ is still the union of an at most countable collection of disjoint segments.
\end{proof}

\end{document}