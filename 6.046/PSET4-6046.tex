\documentclass[psamsfonts]{amsart}

%-------Packages---------
\usepackage{amssymb,amsfonts}
\usepackage[all,arc]{xy}
\usepackage{enumerate}
\usepackage{mathrsfs}
\usepackage[margin=1in]{geometry}
\usepackage{thmtools}
\usepackage{verbatim}
\usepackage{multirow}
\usepackage{tikz}
\usepackage{hyperref}
\usetikzlibrary{shapes,arrows}


%--------Theorem Environments--------
%theoremstyle{plain} --- default
\newtheorem{prob}{Problem}[section]
\newtheorem{thm}{Theorem}[section]
\newtheorem{cor}[thm]{Corollary}
\newtheorem{prop}[thm]{Proposition}
\newtheorem{lem}[thm]{Lemma}
\newtheorem{conj}[thm]{Conjecture}
\newtheorem{quest}[thm]{Question}

\newenvironment{sol}{{\bfseries Solution}}{\qedsymbol}


\theoremstyle{definition}
\newtheorem{defn}[thm]{Definition}
\newtheorem{defns}[thm]{Definitions}
\newtheorem{con}[thm]{Construction}
\newtheorem{exmp}[thm]{Example}
\newtheorem{exmps}[thm]{Examples}
\newtheorem{notn}[thm]{Notation}
\newtheorem{notns}[thm]{Notations}
\newtheorem{addm}[thm]{Addendum}
\newtheorem{exer}[thm]{Exercise}

\theoremstyle{remark}
\newtheorem{rem}[thm]{Remark}
\newtheorem{rems}[thm]{Remarks}
\newtheorem{warn}[thm]{Warning}
\newtheorem{sch}[thm]{Scholium}

\makeatletter
\let\c@equation\c@thm
\makeatother
\numberwithin{equation}{section}

\bibliographystyle{plain}

\voffset = -10pt
\headheight = 0pt
\topmargin = -20pt
\textheight = 690pt

%--------Meta Data: Fill in your info------
\title{6.046 \\
Problem Set 4}

\author{John Wang}

\begin{document}

\maketitle

Collaborators: 

\section{Problem A}

\begin{prob}
Define $f_p: \{0,1\}^m \to \mathrm{Z}$ as $f_p(X) := g(X) \pmod{p}$, where $p$ is a prime and $g:\{0,1\}^{m} \to \mathrm{Z}$ is a function that converts an $m$ bit binary string to a corresponding base 2 integer. Take the set $P = \{p_1, p_2, \ldots, p_t \}$, where $p_i$ are all primes less than some large integer $K$. Suppose we choose a prime $p$ uniformly at random from the set $P$ and take $m$ bit strings $X$ and $Y$ such that $X \neq Y$. Prove that we can bound the probability of a false positive as $P(f_p(X) = f_p(Y)) \leq m/t$. 
\end{prob}

\begin{sol}
We know that $f_p(X) = f_p(Y)$ if and only if $g(X) \equiv g(Y) \pmod{p}$ for some prime $p$ from the set $P = \{p_1, \ldots, p_t \}$. This is equivalent to $|g(X) - g(Y)| \equiv 0 \pmod{p}$. Moreover, since $g(X)$ and $g(Y)$ are both integers, we know that $|g(X) - g(Y)|$ is also an integer. By the Fundamental Theorem of Arithmetic, it can be factored into primes: $|g(X) - g(Y)| = q_1^{e_1} q_2^{e_2} \ldots q_r^{e_r}$ where $q_i$ are prime. However, we know that $r \leq m$ because there are at most $m$ bits in each string.

Moreover, we know that each prime $q_i$ has a $1/t$ chance of being equal to $p$. Since selecting any of the $q_i$ are independent, we see that the probability of $g(X) = g(Y) \pmod{p}$ is just the probability that any of the $q_i$ are equal to $p$. Since there are at most $m$ such $q_i$, each with a probaiblity of $1/t$ of being selected, we see that $P(f_p(X) = f_p(Y)) = m (1/t) = m/t$.   
\end{sol}

\section{Problem B}

\begin{prob}
Design a randomized algorithm that determines if there is a match between a pattern and target text for offset $j$, where $j \in \{ 1,2, \ldots, n - m + 1\}$. 
\end{prob}

\begin{sol}
Choose a random prime from the set $P = \{p_1, p_2, \ldots, p_t \}$ where $P$ consists of all primes less than some large integer $K$. Define $f_p(X)$ as above, and define $\lambda$ as the pattern and $\gamma(j)$ as the target text starting at offset $j \in \{1, 2, \ldots, n-m+1\}$ of length $m$ bits. Now, compute $f_p(\lambda)$, and compute $f_p(\gamma(j))$. If $f_p(\lambda) = f_p(\gamma(j))$, then check to make sure that that $\lambda = \gamma(j)$ by comparing the strings bit by bit. Note that to pick $p$, we don't actually enumerate $P = \{p_1, \ldots, p_t \}$. Instead, we pick a random number less than $K$, and check if it is prime using a primality testing algorithm, and loop until we find a prime. 

To examine the runtime of this algorithm, we first note that there is an algorithm that runs in polynomial time for primality testing. In other words, we can test whether an $b$ bit number is prime in $poly(b)$ time. Since the number of primes less than $k$ is equal to $k / \log k$, the expected number of numbers we have to check is the density of primes, or $k / (k / \log k) = \log k$. The expected running time to find $p$ will then be $\log(k) * poly(\log k) = poly(\log k)$. 

In the worst case, we will have $f_p(\gamma(j)) = f_p(\lambda)$ and the algorithm will have to compare the two strings, which will take $O(m)$ time. The expected worst case run time is just the probability of a false positive, times the time it takes to evaluate the false positive, which is $O(m^2/t)$. One can choose $t$ to be as large as necessary, so this run-time is very small. Notice that if $t = m^2$, then the runtime becomes $O(1)$--this observation will be used in later algorithms. 
\end{sol}

\section{Problem C}

\begin{prob}
Design a formula that given $g(X(j))$ computes $g(X(j+1))$ where $X(j)$ is a length $m$ substring of the target test that starts at position $j$, where $j \in \{1,2,\ldots,n-m+1\}$. Use it to compute $f_p(X(j+1))$ from $f_p(X(j))$. 
\end{prob}

\begin{sol}
If we expand out $g(X(j))$, denoting $x_i$ as the $i$th element in the target string, we obtain:
\begin{eqnarray}
g(X(j)) &=& x_j 2^m + x_{j+1} 2^{m-1} + x_{j+2} 2^{m-2} + \ldots + x_{j+m}
\end{eqnarray}

Therefore, we know that we can obtain $g(X(j+1))$ by using the following manipulations:
\begin{eqnarray}
g(X(j+1)) &=& x_{j+1} 2^m + x_{j+2} 2^{m-1} + \ldots + x_{j+m+1} \\
&=& \left( (x_j 2^m + x_{j+1} 2^{m-1} + \ldots + x_{j+m} - x_{j} 2^m \right)2 + x_{j+m+1} \\
&=& \left( g(X(j)) - x_j 2^{m} \right) 2 + x_{j + m + 1} \\
&=& 2 g(X(j)) - x_j 2^{m+1} + x_{j+m+1}
\end{eqnarray}

Now we can compute $f_p(X(j+1))$ by simplying computing $g(X(j+1))$ and modding it by $p$, since addition and multiplication are preserved under the modulo:
\begin{eqnarray}
f_p(X(j+1)) = 2 f_p(X(j)) - x_j 2^{m+1} + x_{j+m + 1} \pmod {p}
\end{eqnarray}

This completes the formula for deriving $f_p(X(j+1))$ from $f_p(X(j))$. 
\end{sol}

\section{Problem D}

\begin{prob}
Suppose that $X(j)$ and $Y$ differ at every string position. What is the expected number of positions such that $f_p(X(j)) = f_p(Y)$?
\end{prob}

\begin{sol}
We use the probability bound we derived in part $a$, namely that if $X \neq Y$, then $P(f_p(X) = f_p(Y)) \leq m/t$ and define $M$ as a random variable of the number of positions such that $f_p(X(j)) = f_p(Y)$ for $j \in \{1,2,\ldots,n-m+1\}$. Also define $M_i$ as indicator random variables for whether or not there are $i$ positions $\{j_1, j_2, \ldots, j_i\}$ such that $f_p(X(j_i)) = f_p(Y)$. We can derive a worst case expected number of positions as:
\begin{eqnarray}
\mathbb{E}[M] &=& \sum_{i=0}^{n-m} M_i P(M_i = 1) \\
&=& \sum_{i=0}^{n-m} M_i P(f_p(X(j_i)) = f_p(Y)) \\
&\leq& \sum_{i=0}^{n-m} \frac{m}{t} \\
&=& \frac{m(n-m + 1)}{t}
\end{eqnarray} 

We see that $\mathbb{E}[M] = O \left(\frac{m(n-m)}{t} \right) = O \left( \frac{mn}{t} \right) $. 
\end{sol}

\section{Problem E}

\begin{prob}
Using parts above, design a randomized algorithm that determines if there is a match between a pattern and a target text in $O(n +m)$ expected running time. The algorithm should always return the correct answer.
\end{prob}

\begin{sol}
Consider the following algorithm. First, all primes in $P = \{p_1, \ldots, p_t \}$ are less than some integer $k$, which will be chosen such that $m^2 = \frac{k}{\log k}$. Note that we can compute $k$ in $O(\log m^2) = O(2 \log m) = O(\log m)$ time using Newton's Method. To use Newton's method, we define $f(k) = m^2 - k/\log(k)$ so that we can find a root of $f(k)$ iteratively using Newton's method of $k_{i+1} = k_{i} - f(k_i)/f'(k_i)$. We have the following recursive formula to compute $k$:
\begin{eqnarray}
k_{i+1} = k_i - \frac{(m^2 \log k - k) \log k}{\log k - 1}
\end{eqnarray}

This works and has quadratic convergence as long as $k \neq 0$, which should not happen since $n > 0$. Quadratic convergence follows because $f(k)$ is continuously differentiable for $k > 0$, $f'(\alpha) \neq 0$ at the root $\alpha$, and $f''(\alpha)$ exists.\footnote{This follows from a theorem on Wikipedia, whose proof is given here: \url{http://en.wikipedia.org/wiki/Newton's_method\#Proof\_of\_quadratic\_convergence\_for\_Newton.27s\_iterative_method}.} Since the convergence is quadratic in $n$, we know that we can find $k$ in $O(\log n)$ time. 

Next, we randomly select $p \in \{p_1, p_2, \ldots, p_t \}$ such that for each $p_j$, we have $p_j < k$. This can be done in $O(1)$ time as in the directions of the Pset. Now, we compute $f_p(X(j))$ for all $j \in \{1,2,\ldots,n-m+1\}$. We can do this by first computing $f_p(X(0))$ by using Horner's method to compute $t_1 2^{m} + t_2 2^{m-1} + \ldots + t_m$ using:
\begin{eqnarray}
t_1 2^{m} + t_2 2^{m-1} + \ldots + t_m = t_m + 2(t_{m-1} + 2(t_{m-2} + \ldots + 2(t_{2} + 2 t_1) \ldots ) ) 
\end{eqnarray}

Next, we can compute $f_p(X(1))$ by applying the transformation from problem C. We use this recursively to compute $f_p(X(j+1))$ from $f_p(X(j))$ for all the $j$. Next, we compute $f_p(Y)$ using Horners method to compute $p_1 2^{m} + \ldots + p_m$ in the same way we computed $f_p(X(0))$. Next, we iterate through $f_p(X(j))$ for all $j \in \{1,2,\ldots,n-m+1\}$. If $f_p(X(j)) = f_p(Y)$, then check whether the bits $t_{j}$ through $t_{j+m}$ match the bits in the pattern string $Y$. If the bits match, then append this $j$ to the output. If the bits do not match, do nothing. Continue looping through all $j$, and return all the matches that have been found.

First, we start with proof of correctness. We know that if $X(j) = Y$, then we must have $f_p(X(j)) = f_p(Y)$ for any $p \in \{p_1, p_2, \ldots, p_t \}$. This follows directly from problem $a$. Next, we know that if we check the bits of $X(j)$ with $Y$, then we can be sure whether $X(j)$ is the same as $Y$ with certainty. Therefore, if $X(j) = Y$ for any $j$, then the algorithm will correctly identify that $f_p(X(j)) = f_p(Y)$ and then that $X(j) = Y$ by checking the bits. Next, we need to prove that the algorithm will not return a false positive. Comparing the bits in $X(j) = Y$ returns a positive output if and only if $X(j) = Y$. Therefore, the algorithm will never return a false positive since we will always check the bits. Therefore, the algorithm must return a correct output.

Next, we will analyze the runtime. We can compute $k$ in $O(\log m)$ time, as mentioned in the presentation of the algorithm. We can compute $f_p(X(0))$ using Horner's rule in $O(m)$ time, since $X(0)$ is a polynomial of degree $m$. Computing all of the $f_p(X(j))$ for $j \in \{1, 2, \ldots n-m+1\}$ requires $O(n-m)$ time since computing $f_p(X(j+1))$ from $f_p(X(j))$ requires $O(1)$ arithmetic operations, each costing $O(1)$ time. Therefore, the total time for computing $f_p(X(j))$ for $j \in \{0,1,\ldots,n-m+1\}$ takes $O(n+m)$ time. Each comparison of $f_p(X(j))$ with $f_p(Y)$ requires $O(1)$ time, so it takes $O(n-m) = O(n)$ to compare all the $f_p(X(j))$. If $M$ is the number of matches where $f_p(X(j)) = f_p(Y)$, and therefore the number of comparisons required. The rest of the algorithm requires $O(Mm)$ time, since each comparison of two $m$ bit strings requires $O(m)$ time. Thus, the running time is $T(n,m) = O(m) + O(\log m) + O(n-m) + O(n+m) + O(n) + O(Mm) = O(n+m) + O(Mm)$. To find the expected worst case running time, we invoke the results from part $d$ about $\mathbb{E}[M]$, the expected number of positions such that $f_p(X(j)) = f_p(Y)$:
\begin{eqnarray}
\mathbb{E}[T(n,m)] &=& \mathbb{E}[O(n+m) + O(Mm)] \\
&=& O(n+ m) + \mathbb{E}[M] O(m) \\
&=& O(n+m) + O \left( \frac{m^2(n-m)}{t} \right) 
\end{eqnarray}

Where we have substituted $E[M] = O \left( \frac{m(n-m)}{t} \right)$ from part $d$. Next, we know that $t \sim \frac{k}{\log k}$. Since we have set $k$ such that $m^2 = \frac{k}{\log k}$, we see that $ t \sim m^2$. This means that our expected runtime becomes:
\begin{eqnarray}
\mathbb{E}[T(n,m)] &=& O(n+m) + O \left( \frac{m^2(n-m)}{m^2} \right) \\
&=& O(n+m) + O(n) \\
&=& O(n+m)
\end{eqnarray}

Now, we will go through an example of the algorithm. Let us have the pattern $Y=2 = 10_{2}$ and the text string $X = 5 = 101_{2}$. First, we note that $m = 2$ and $n=2$. We pick $k$ such that $m^2 = 4 = k / \log k$. A close approximation is $k = 9$, which would have been computed using Newton's method. Now, we pick a random prime less than $9$. Let us pick $p = 3$. Now, we compute $f_p(X(0))$ and $f_p(X(1))$. There are $f_p(X(0)) = 2 \pmod{3}$ and $f_p(X(1)) = 1 \pmod{3}$. We also compute $f_p(Y) = 2 \pmod{3}$. We see that $f_p(X(0))$ matches $f_p(Y)$, and $f_p(X(1))$ does not. So we check to see if the bits of $X(0)$ match the bits of $Y$. Indeed, they are both $10$, so the algorithm returns $j = 0$ as the only output. We can see by inspection that this is the correct output.  
\end{sol}

\section{Problem f}

\begin{prob}
Provide a bound for the probability that the running time is more than 100 times the expected running time.
\end{prob}

\begin{sol}
Markov's inequality states that $P[X > c \mathbb{E}[X] ] \leq \frac{1}{c}$ where $c$ is a constant and $X$ is a non-negative random variable. Since $T(n,m)$ is a non-negative random varialbe (the runing time can never be negative, we can apply Markov's inequality to $\mathbb{E}[T(n,m)]$. We observe:
\begin{eqnarray}
P\left[T(n,m) > 100 \mathbb{E}[T(n,m)] \right] \leq \frac{1}{100}
\end{eqnarray}

Thus, the probability that the running time is more than 100 times the expected running time is less than $1/100$.
\end{sol}

\end{document}