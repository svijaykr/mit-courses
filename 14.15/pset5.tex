\documentclass[psamsfonts]{amsart}

%-------Packages---------
\usepackage{amssymb,amsfonts}
\usepackage{enumerate}
\usepackage[margin=1in]{geometry}
\usepackage{amsthm}
\usepackage{theorem}
\usepackage{verbatim}
\usepackage{graphics}
\usepackage{float}

\newenvironment{sol}{{\bfseries Solution:}}{\qedsymbol}
\newenvironment{prob}{{\bfseries Problem:}}

\bibliographystyle{plain}

\voffset = -10pt
\headheight = 0pt
\topmargin = -20pt
\textheight = 690pt

%--------Meta Data: Fill in your info------
\title{14.15 \\
Networks \\
Problem Set 5}

\author{John Wang}

\begin{document}

\maketitle

Collaborators: Ryan Liu, Albert Wu

\section{Problem 1}

\begin{prob}
  Sudoku as a game of cooperation
\end{prob}
\begin{sol}
  We will use the potential function $- \Phi^*(\alpha) = \Phi(\alpha) = \sum_{i \in I} u_i(\alpha)$ where $I$ is the set of all agents playing the game. We shall show that this is a valid potential function for the game, making the game a potential game.

  First, let us suppose that agent $j$ chooses between move $x$ and $z$ for one of his squares. In this case, the potential difference is given by $\Phi(x, \alpha^*) - \Phi(z, \alpha^*)$ where $\alpha^*$ is the vector of choices for the agents for all but the square corresponding to move $x$ and $z$. We have:
  \begin{eqnarray}
    \Phi(x, \alpha^*) - \phi(z, \alpha^*) &=& \sum_{i \in I} u_{i} (x, \alpha^*) - u_{i}(z, \alpha^*) \\
                                          &=& \sum_{i \in I} \left( n_i^R(x, \alpha^*) + n_i^C(x, \alpha^*) + n_i^B(x, \alpha^*) \right) \\ 
                                          &-& \left( n_i^R(z, \alpha^*) + n_i^C(z, \alpha^*) + n_i^B(z, \alpha^*) \right) \nonumber
  \end{eqnarray}

  However, we know that $n_i^R(x, \alpha^*) - n_i^R(z, \alpha^*) = 0$ except when $i$ is in the same row as the move that $j$ makes. Likewise, the difference between column and block sums are zero when the columns and blocks do not contain the block that player $j$ is changing with moves $x$ and $z$.

  This means that we can simplify the above expression to:
  \begin{eqnarray}
    \Phi(x, \alpha^*) - \phi(z, \alpha^*)&=& \sum_{i \in R_j} n_i^R(x, \alpha^*) - n_i^R (z, \alpha^*) \\
                                         &+& \sum_{i \in B_j} n_i^B(x, \alpha^*) - n_i^B(z, \alpha^*) + \sum_{i \in C_j} n_i^C(x, \alpha^*) - n_i^C(z, \alpha^*) \nonumber
  \end{eqnarray}

  Note, however, that for each row $i \in R_j$, the number of repetitions in the row remains the same for each square in the row. This means that $n_i^R(x, \alpha^*) = n_k^R(x, \alpha^*$ for $i, k \in R_j$. The same is true for blocks and columns. This means that we can simplify the above expression to:
  \begin{eqnarray}
    \Phi(x, \alpha^*) - \phi(z, \alpha^*) &=& 8 (n_i^R(x, \alpha^*) - n_i^R (z, \alpha^*)) + 8(n_i^C(x, \alpha^*) - n_i^C (z, \alpha^*)) + 8 (n_i^B(x, \alpha^*) - n_i^B (z, \alpha^*))
  \end{eqnarray}

  However, since we have defined the utility as $u_i(\alpha) = - (n_i^R(\alpha) + n_i^C (\alpha) + n_i^B (\alpha))$. This means that our potential function has the following property:
  \begin{eqnarray}
    \Phi^*(x, \alpha^*) - \Phi^*(z, \alpha^*) \geq 0 \textrm{ iif } u_i(x, \alpha^*) - u_i(z, \alpha^*)
  \end{eqnarray}

  This shows that the game is a potential game.
\end{sol}

\newpage

\section{Problem 2}

\begin{prob}
  Sequential duel
\end{prob}
\begin{sol}
  The sequential duel can be modelled as a sequential form game with two players $x,y$. We shall denote strategies on the $k$ stage of the duel as $s_i^k: H^k \to s_i$, where $H^k$ is the set of all possible stage-$k$ histories and $s_i$ is the set of all strategies. We have $s_i^k \in \{ \textrm{shoot}, \textrm{don't shoot} \}$. 

  The utility for person $i$ at stage $k$ for strategy $s_j$ is given by $u_i^k(s_j)$:
  \begin{eqnarray}
    u_i^k(s_j) = \left\{ \begin{array}{ l l}
        \pi_i^{k} (1 - p_j) & \textrm{if } s_j^k = \textrm{ shoot} \\
                  \pi_i^{k} & \textrm{else}
      \end{array}
      \right.
  \end{eqnarray}

  Where we define $\pi_i^k$ as the probability of person $i$ surviving at stage $k$.

  Now we shall show that no one shooting is a subgame perfect equilibrium. If both players have this strategy, then $\lim_{k \to \infty} \pi_i^{k} = 1$ and both players are guaranteed to survive. It is clear that this is a subgame perfect equilibrium because any deviation is not advantageous and cannot improve the outcome (since the outcome is already guaranteed to be the best possible). Any deviation, however, cause cause the other player to change their strategy in response, causing a probability of surviving to be lower, which is not optimal.

  We shall now show that shooting with probability 1 is a subgame perfect equilibrium. Suppose $y$ shoots with probability $1$. Then we shall show that $x$ has no better deviation from shooting with probability $1$. Since the game is symmetric, this is show the strategy is subgame perfect.  Note that if $x$ shoots in this stage, then $y$ has probability $\pi_i^{k-1} (1 - p_j)$ of living in stage $k+1$. If $x$ does not shoot, then $y$ has probability $\pi_i^{k-1} > \pi_i^{k-1} (1 - p_j)$ of living. Since $x$ has a higher probability of killing $y$ in this stage and therefore surviving (since $y$ shall not change his strategy), then $x$'s best chance is to shoot. Therefore, shooting with probability 1 is an equilibrium.
\end{sol}

\section{Problem 3}

\begin{prob}
  Choosing the estate tax rate as an extensive form game
\end{prob}
\begin{sol}

\end{sol}

\section{Problem 4}

\subsection{Problem 4.a}

\begin{prob}
Characterize a threshold value of this discount factor such that above this threshold, there exists a subgame perfect equilibrium in which there is a collusive equilibrium in which both firms charge a price equal to $R$ at each date.
\end{prob}
\begin{sol}
  First, we note that in order to have a subgame perfect equilibrium on an infinitely-repeated game, we must have the same strategy repeated in each stage. In the context of a symmetric Bertrand competition game, this means that both firms must change the same price $R$.

  For this to be true, it must be disadvantageous to deviate from $R$. Now, let us consider firm $x$'s decision to deviate. If $x$ deviates, then he will gain all of the market in the first stage, but he may forgo all the gains he would get in later stages because the second firm may outprice him. Thus, the payoff from deviating in the first stage must be greater than the payoff of splitting the demand with price $R$.

  Let $\pi = (R - c) D(R)$ be the profit if one firm has the market at price $R$. We must therefore have:
  \begin{eqnarray}
    \pi &\leq& \frac{\pi}{2} (1 + \delta + \delta^2 + \ldots) \\
        &=& \frac{1}{1- \delta} \frac{\pi}{2} \\
  \end{eqnarray}

  Thus, we see that the discount rate must be $\delta \geq \frac{1}{2}$.
\end{sol}

\subsection{Problem 4.b}

\begin{prob}
  Now suppose that the two firms have different discount factors $\delta_1 < \delta_2$. When is it subgame perfect equilibrium for both firms to charge a price of $R$ at each date?
\end{prob}
\begin{sol}
  In order for it to be a subgame perfect equilibrium, it must be the case that deviating and losing all future profits while gaining the first stage's profits in totality is not as good as not deviating. Writing this out, we have the following conditions:
  \begin{eqnarray}
    (R - c) D(R) &\leq& \frac{(R - c) D(R)}{2} ( 1 + \delta_1 + \delta_1^2 + \ldots) \\
    (R - c) D(R) &\leq& \frac{(R - c) D(R)}{2} ( 1 + \delta_2 + \delta_2^2 + \ldots) \\
  \end{eqnarray}

  Using the series representation of $\delta_1$ and $\delta_2$, both inequalities simplify to:
  \begin{eqnarray}
    (1 - \delta_1) (R - c) D(R) &\leq& \frac{(R - c) D(R)}{2} \\
    (1 - \delta_2) (R - c) D(R) &\leq& \frac{(R - c) D(R)}{2}
  \end{eqnarray}

  Adding the inequalities together and simplifying further, we find:
  \begin{eqnarray}
    (1 - \delta_1) + (1 - \delta_2) &\leq& 1 \\
                \delta_1 + \delta_2 &\geq& 1
  \end{eqnarray}

  Which is the condition we are looking for.
\end{sol}

\subsection{Problem 4.c}

\begin{prob}
  Now suppose that the marginal cost of production are $c_1 < c_2 < R$, and both firms have the same discount factor $\delta$. When is it a subgame perfect equilibrium for both for the charge $R$ at each date?
\end{prob}
\begin{sol}
  Again, in order for this equilibrium to occur, it must be the case that deviating cannot increase the payoff of any actor. Thus, the payoff of winning a single stage is less than the payoff of halving all the stages. This can be written as:
  \begin{eqnarray}
    (R - c_1) D(R) &\leq& \frac{(R - c_1) D(R)}{2} ( 1 + \delta + \delta^2 + \ldots) \\
    (R - c_2) D(R) &\leq& \frac{(R - c_2) D(R)}{2} ( 1 + \delta + \delta^2 + \ldots) \\
  \end{eqnarray}

  We divide both sides by $(R - c_i) D(R)$ to obtain the following (repeated) condition:
  \begin{eqnarray}
    1 &\leq& \frac{1}{2} \frac{1}{1 - \delta} \\
    \frac{1}{2} &\leq& \delta
  \end{eqnarray}
\end{sol}
\end{document}


