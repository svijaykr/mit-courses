\documentclass[psamsfonts]{amsart}

%-------Packages---------
\usepackage{amssymb,amsfonts}
\usepackage[all,arc]{xy}
\usepackage{enumerate}
\usepackage{mathrsfs}
\usepackage[margin=1in]{geometry}
\usepackage{thmtools}
\usepackage{verbatim}
\usepackage{multirow}


%--------Theorem Environments--------
%theoremstyle{plain} --- default
\newtheorem{prob}{Problem}[section]
\newtheorem{thm}{Theorem}[section]
\newtheorem{cor}[thm]{Corollary}
\newtheorem{prop}[thm]{Proposition}
\newtheorem{lem}[thm]{Lemma}
\newtheorem{conj}[thm]{Conjecture}
\newtheorem{quest}[thm]{Question}

\newenvironment{sol}{{\bfseries Solution}}{\qedsymbol}


\theoremstyle{definition}
\newtheorem{defn}[thm]{Definition}
\newtheorem{defns}[thm]{Definitions}
\newtheorem{con}[thm]{Construction}
\newtheorem{exmp}[thm]{Example}
\newtheorem{exmps}[thm]{Examples}
\newtheorem{notn}[thm]{Notation}
\newtheorem{notns}[thm]{Notations}
\newtheorem{addm}[thm]{Addendum}
\newtheorem{exer}[thm]{Exercise}

\theoremstyle{remark}
\newtheorem{rem}[thm]{Remark}
\newtheorem{rems}[thm]{Remarks}
\newtheorem{warn}[thm]{Warning}
\newtheorem{sch}[thm]{Scholium}

\makeatletter
\let\c@equation\c@thm
\makeatother
\numberwithin{equation}{section}

\bibliographystyle{plain}

\voffset = -10pt
\headheight = 0pt
\topmargin = -20pt
\textheight = 690pt

%--------Meta Data: Fill in your info------
\title{18.781 \\
Problem Set3}

\author{John Wang}

\begin{document}

\maketitle

\section{Problem 1}

\begin{prob}
Solve the congruence $x^3 - 9x^2 + 23x - 15 \equiv 0 \pmod{143}$. 
\end{prob}

\begin{sol}
First, we note that the prime factorization of $143 = (11)(13)$. Thus, we can take the congruence modulo $11$ and $13$ to create two new congruences, since $11$ and $13$ are prime. This yields:
\begin{eqnarray}
x^3 + 2x^2 + x + 7 &\equiv& 0 \pmod{11} \\
x^3 + 4x^2 + 10x + 11 &\equiv& 0 \pmod{13} 
\end{eqnarray} 

In the first equation, we can check all $x \in \{0, 1, \ldots, 10 \}$, and we find that the congruence holds for $x=1,3,5$. We can do the same in the second equation, and we see that the congruence also holds for $x = 1,3,5$. Thus, we want to find solutions to the simultaneous equations $x \equiv a \pmod{11}$ and $x \equiv b \pmod{13}$ where $a \in \{1,3,5\}$ and $b \in \{1,3,5\}$. Now we can use the Chinese Remainder Thereom to determine solutions $x(a,b)$ to each of these simultaneous equations. We know that the solution is $x = N_a H_a a + N_b H_b b$ where $N_a = m_b$, $N_b = m_a$, $H_a$ is the multiplicative inverse of $N_a \pmod {a}$, and $H_b$ is the multiplicative inverse of $N_b \pmod{b}$. Therefore, we know that $N_a = 11$, $N_b = 13$, $H_a = 6$, and $H_b = 6$. This means that $x(a,b) = 66a + 78b \pmod{143}$. We can list the solutions then:
\begin{eqnarray}
x(1,1) &=& 66(1) + 78(1) = 144 \equiv 1 \pmod{143} \\
x(1,3) &=& 66(1) + 78(3) = 300 \equiv 14 \pmod{143} \\
x(1,5) &=& 66(1) + 78(5) = 456 \equiv 27 \pmod{143} \\
x(3,1) &=& 66(3) + 78(1) = 276 \equiv 133 \pmod{143} \\
x(3,3) &=& 66(3) + 78(3) = 432 \equiv 3 \pmod{143} \\ 
x(3,5) &=& 66(3) + 78(5) = 588 \equiv 16 \pmod{143} \\
x(5,1) &=& 66(5) + 78(1) = 408 \equiv 122 \pmod{143} \\
x(5,3) &=& 66(5) + 78(3) = 564 \equiv 135 \pmod{143} \\
x(5,5) &=& 66(5) + 78(5) = 720 \equiv 5 \pmod{143}
\end{eqnarray}

Thus we have the following solutions: $x = 1,14,27,133,3,15,122,135,5 \pmod{143}$. 
\end{sol}

\section{Problem 2}

\begin{prob}
What are the last two digits of $2^{100}$ and $3^{100}$?
\end{prob}

\begin{sol}
First notice that we want to find $2^{100} \pmod{100}$ in order to obtain the last two digits of $2^{100}$. Also notice that the sequence $2^{k} \pmod{100}$ repeats itself in a cycle of length $20$. We see that $2^2 \equiv 4 \pmod{100}$ and $2^{22} \equiv 4 \pmod{100}$. Therefore, it is clear that $2^{{22 + 3*20}} \equiv 2^{82} \equiv 4 \pmod{100}$. We also know that $2^{100} = 2^{82} 2^{18} = 2^{82} 2^{10} 2^{8} \equiv (4)(24)(56) \pmod{100} \equiv 76 \pmod{100}$. Therefore, the last two digits in $2^{100}$ are $76$. 

To find the last two digits in $3^{100}$, we want to obtain $3^{100} \pmod{100}$. We will use a similar process. As it turns out, the sequence $3^{k} \pmod{100}$ repeats itself in a cycle of length $20$, just as it does for the powers of $2$. One can see that $3^2 \equiv 9 \pmod{100}$ and that $3^{22} \equiv 9 \mod{100}$. Therefore, we see that $3^{22 + 3*20} \equiv 3^{2} \pmod{100}$. We can also write $3^{100} = 3^{82} 3^{10} 3^{8} \equiv (9)(49)(61) \pmod{100}$ since $3^{10} = 59049$ and $3^8 = 6561$. Thus we see that $3^{100} \equiv 1 \pmod{100}$. This the last two digits of $3^{100}$ are $01$.  
\end{sol}

\section{Problem 3}

\begin{prob}
Find the number of solutions of $x^2 \equiv x \pmod{m}$ for any positive integer $m$. 
\end{prob}

\begin{sol}
First, we know that the $x^2 \equiv x \pmod{m}$ is equivalent to $x^2 - x = x (x - 1) \equiv 0 \pmod{m}$. First, if $m$ is prime, then we know there are less than or equal to 2 solutions. However, we can choose $x = 0$ and $x = 1$ which will satisfy the congruence. Thus, if $m$ is prime, there will be two solutions. 

Now, suppose $m$ is composite. Then it can be decomposed into its prime factors: $m = p_1^{e_1} p_2^{e_2} \ldots p_r^{e_r}$. If $x$ satisfies $x(x-1) \equiv 0 \pmod{m}$ then $m | x(x-1)$ so that $p_i^{e_i} | x(x-1)$. This means that $x(x-1) \equiv 0 \pmod{p_i^{e_i}}$. Moreover, if $x_i$ is a solution of $x(x-1) \equiv 0 \pmod{p_i^{e_i}}$ for all $i \in \{1, \ldots, r\}$, then there exists a unique $x \pmod{m}$ such that:
\begin{eqnarray}
x &\equiv& x_1 \pmod{{p_1}^{e_1}} \\
&\vdots& \\
x &\equiv& x_r \pmod{{p_r}^{e_r}}
\end{eqnarray}
This follows since we know that $p_1^{e_1}, p_2^{e_2}, \ldots, p_r^{e_r}$ are relatively prime in pairs and we can therefore apply the Chinese Remainder Theorem. Now $x$ satisfies $x(x-1) \equiv 0 \pmod{{p_i}^{e_i}}$. This shows there exists a bijection where the number of solutions to $x^2 \equiv x \pmod{m}$ is equal to the number of solutions to product of the number of solutions to $x^2 \equiv x \pmod{p_i^{e_i}}$ for all $i$. Thus, if we denote $N(m)$ as the number of solutions $x(x-1)$ modulo $m$. Thus, we see that $N(m) = N(p_1^{e_1}) \ldots N(p_r^{e_r})$. 

Now, to evalutate $N(p_i^{e_i})$, we must find the solutions to $x(x-1) \equiv 0 \pmod{p_i}^{e_i}$. However, we know that for any integer $x > 1$, we must have $(x, x-1) = 1$. Therefore, it must be true that if $p | x$ then $p \nmid x - 1$ and conversely if $p | x- 1$ then $p \nmid x$. Therefore, either $p | x$ or $p | x - 1$. Moreover, we know that there are at most 2 solutions because $x^2 - x$ is of degree 2 and $p_i^{e_i}$ is a multiple of a prime. Moreover, since $x = 0, 1 \pmod {p_i}^{e_i}$ always work, we know there are exactly 2 solutions. Thus, if $r$ is the number of unique prime factors of $m$, there will be $2^r$ solutions to $x^2 \equiv x \pmod{m}$. Since this works when $m$ is prime and when is composite, we are done and there are $2^r$ solutions. 
\end{sol}

\section{Problem 4}

\begin{prob}
Show that the number $n=561 = 3 \cdot 11 \cdot 17$ satisfies the property $P$: for any $a$ coprime to $n$, we have $a^{n-1} \equiv 1 \pmod{n}$.
\end{prob}

\begin{sol}
We must show that $a^{n-1} \equiv 1 \pmod{n}$ or equivalently that $a^{n} \equiv a \pmod{n}$. First, we know that $3,11,17$ are all primes so that they are pairwise coprime. This means that $a^{\phi(p)} \equiv 1 \pmod{p}$, for any $p \in \{3,11,17\}$ by Euler's Theorem. We know that for a prime $\phi(p) = p - 1$, so that $\phi(3) = 2$, $ \phi(11) = 10$, and $\phi(17) = 16$. Moreover, we see that $(a, 3) = (a,11) = (a,17) = 1$ since $3,11,17$ are prime. Writing this out, we have:
\begin{eqnarray}
a^2 &\equiv& 1 \pmod{3} \\
a^{10} &\equiv& 1 \pmod{11} \\
a^{16} &\equiv& 1 \pmod{17} 
\end{eqnarray}

Since $a^{560} = (a^2)^{280} = (a^{10})^{56} = (a^{16})^{35}$, we see that $a^{560} \equiv 1 \pmod{3}$, $a^{560} \equiv 1 \pmod{11}$, and $a^{560} \equiv 1 \pmod{17}$. However, since $3,11,17$ are coprime, the system has a unique solution with modulus $3 \cdot 11 \cdot 17 = 561$ by the Chinese Remainder Theorem. Therefore, we see that $a^{560} \equiv 1 \mod{561}$, which is what we wanted.
\end{sol}

\begin{prob}
Let $n$ be a squarefree composite number satisfying P. Show that $n$ has at least 3 prime factors.
\end{prob}

\begin{sol}
First we will prove a small lemma which will allow us to reach a contradiction.

\begin{lem}
If $n$ is a squarefree composite number satisfying P and $p$ is a prime factor of $n$, then $p-1 | n - 1$. 
\end{lem}

\begin{proof}
Since $n$ is a squarefree composite number satisfying P, we know that $n | a^n - a$ for all $a$ such that $(a,n) = 1$. This means that $p | a^n - a$ since $p$ is a prime factor of $n$. However, we know that $p \nmid a$ since $(a,p) = 1$ so that $p | a (a^{n-1} - 1)$ implies $p | a^{n-1} - 1$. This shows that $a^{n-1} \equiv 1 \pmod{p}$, which means that $n-1$ must be a multiple of $\phi(p) = p - 1$ by Fermat's Little Theorem. Thus, we see that $p - 1 | n - 1$. 
\end{proof}

Now tht we have established this lemma, we go back to our problem. If $n$ is a squarefree composite number, then there are at least $2$ prime factors. Let us assume by contradiction that there exist only $2$ prime factors, let them be $p$ and $q$ (we cannot have $n = p^{e_p} q^{e_q}$ where $e_p, e_q > 1$ because we assumed they are squarefree. Assume without loss of generality that $p > q$. Then we use the above lemma to see that
\begin{eqnarray}
p - 1 &|& pq - 1 \\
p - 1 &|& p(q-1) + p - 1 \\
p - 1 &|& p(q-1) 
\end{eqnarray}

Since clearly $p-1 \nmid p$, we must have $p-1 | q-1$. This is a contradiciton because $p - 1 > q - 1$ and $p,q$ are prime. This completes the proof 
\end{sol}

\begin{prob}
Write down a sufficient condition for $n = pqr$ where $p,q,r$ are primes to satisfy property P. Then write a gp program to generate a list of ten such numbers $n$. 
\end{prob}

\begin{sol}
We will show that if $p = 6m + 1$, $q = 12m + 1$, and $r = 18m + 1$ are all prime, then $n=pqr$ satisfies property P. First, we know that if $p,q,r$ are all prime, then $a^{\phi(p)} = a^{p - 1} \equiv 1 \pmod{p}$ so that we have the following:
\begin{eqnarray}
a^{6m} &\equiv& 1 \pmod{p} \\
a^{12tm} &\equiv& 1 \pmod{q} \\
a^{18m} &\equiv& 1 \pmod{r} 
\end{eqnarray}

We know that $p,q,r$ are pairwise coprime because they themselves are primes. Thus, if we can show that $6 | n - 1$, $12|n - 1$, and $18 |n - 1$, then we can show that $a^{n - 1} \equiv 1 \mod{p,q,r}$ and so that the system has a unique solution modulus $n=pqr$ by the Chinese Remainder theorem. This would show $a^{n-1} \equiv 1 \pmod{n}$ and complete the proof. Thus, we must show that $6, 12, 18 | n - 1 = (6m + 1)(12m + 1)(18m + 1) - 1$. Expanding out $n - 1$, we have $n - 1 = 1296 m^3 + 396 m^2 + 36m + 1 - 1 = 1296 m^3 + 396 m^2 + 36m$.  It is clear by inspection that $lcm(6,12,18) = 36$ so that $6|36$, $12|36$, and $18|36$. Since $36 | 396$ and $36 | 1296$, we see that $6m$, $12m$, and $18m$ all divide $n-1$. This completes the proof.  

The following gp code returns a list of the first ten such numbers:
\begin{verbatim}
    count = 0;
    m = 1;
    while(count<10,
        p = 6*m+1;
        q = 12*m+1;
        r = 18*m+1;
        if (isprime(p) && isprime(q) && isprime(r), 
            print(p*q*r);
            count ++;
            );
        m ++;
    )
\end{verbatim}

The resulting output is:
\begin{center}
\begin{tabular}{l}
1729 \\
294409 \\
56052361 \\
118901521 \\
172947529 \\
216821881 \\
228842209 \\
1299963601 \\
2301745249 \\
9624742921 
\end{tabular}
\end{center}
\end{sol}

\section{Problem 5}

\begin{prob}
Do there exist arbitrarily long sequences of consecutive integers, none of which are squarefree? 
\end{prob}

\begin{sol}
We will show that these arbitrarily long sequences do exist. Assume by contradiction that there exists some $k$ such that all sequences $n, n+1, \ldots, n+k$ contain a squarefree number for all $n \in \mathrm{Z}$. Then it is clear that at least every $k$th integer must be squarefree. Now choose the first $m$ primes $p_1, p_2, \ldots, p_m$ such that $p_1 p_2 \ldots p_m + k < p_1 \ldots p_{m-1} p_{m+1}$. We shall show that for any $k$, we can always select an $m$ such that this holds. 

This is because for any $k$, we can always choose $m$ such that $k < (p_{m+1} - p_{m})(p_1 \ldots p_{m-1})$ since there are infinitely many primes, and $(p_1 \ldots p_{m-1})$ can be made arbitrarily large. This means that the following holds:
\begin{eqnarray}
\frac{k}{p_1 \ldots p_{m-1}} &<& p_{m+1} - p_m \\
p_m + \frac{k}{p_1 \ldots p_{m-1}} &<& p_{m+1} \\
p_1 p_2 \ldots p_{m-1} p_{m} + k &<& p_1 \ldots p_{m-1} p_{m+1} 
\end{eqnarray}

Now let us fix $m$ so that $p_1 \ldots p_m + k < p_1 \ldots p_{m-1} p_{m+1}$. First, we know that $n = p_1 \ldots p_m$ must be a squarefree integer. Thus, as we have shown before, there must be another squarefree integer within the next $k$ integers. Thus, we see that $\exists x$ for which $n < x \leq n + k$ such that $x$ is squarefree. However, we also know that in order for $x$ to be squarefree, it must have a prime factorization without any exponents. Moreover, we know that $x > n$. The smallest possible option for $x$ is then $x = p_1 \ldots p_{m-1} p_{m+1}$ because we must replace $x_{m+1}$ with one of the prime factors of $n$ (so as to not repeat a prime factor). However, we have just shown that $p_1 \ldots p_m + k = n + k < p_1 \ldots p_{m+1} p_{m+1} \leq x$. Thus, $n + k < x$, which is a contradiction of the fact that $n < x \leq n + k$. This completes the proof.
\end{sol}

\section{Problem 6}

\begin{prob}
Let $f(x) = x^3 - 2$. Write a gp program to calculate the set $S$ of primes $p$ less than 10000 such that $f$ has a solution modulo $p$. Make a conjeture about the density of such primes.
\end{prob}

\begin{sol}
The gp code is given below. We used the first 1229 primes, since the number of primes below 10000 is 1229. This can be calculated in gp by using $primepi(10000) = 1229$. 

\begin{verbatim}
    numprimes = 1229;
    checkprimes = primes(numprimes);
    solution_count = 0;
    for(i=1, numprimes,
        cprime = checkprimes[i];
        for (j=1, cprime,
            if (Mod(j^3 - 2, cprime) == Mod(0, cprime), 
                solution_count ++;
                break;
            );
        );
    );
    print("Density: ", solution_count/numprimes);
\end{verbatim}

We obtained 818 primes in the set $S$ out of the 1229 possible primes. The resulting density was $0.665$ for the first 1229 primes. Using the first 3000 primes, we see that the density was $0.668$. Thus, I conjecture that the actual density approaches $2/3$ in the limit as $p \to \infty$. 
\end{sol}

\begin{prob}
Now do the same exercise for $f(x) = x^3 - 3x - 1$. 
\end{prob}

\begin{sol}
The only thing that needs to be changed in the previous code is in the $if$ statement. Instead of using $Mod(j^2 - 2, cprime)$, we will now use $Mod(j^3 - 3j - 1, cprime)$. Running the same gp code with this change shows that there are 405 primes less than 10000 for which there exists a solution to $f(x) = x^3 - 3x - 1 \pmod{p}$. Thus the density is $0.329$ for the first 1229 primes. Of the first 3000 primes, 1005 had solutions to $f(x) \equiv 0 \pmod{p}$. Thus, I conjecture that the density approaches $1/3$ in the limit as $p \to \infty$.
\end{sol}

\begin{prob}
What qualitative feature of $f$ differentiates these cases?
\end{prob}

\begin{sol}
The polynomial $x^3 - 3x - 1$ can be factored into $x(x^2 - 3) - 1$ while $x^3 - 2$ cannot be factorized further. Thus, the first case has equivalent form of $x (x^2 - 3) \equiv 1 \pmod{p}$ while the second case only has the form $x^3 \equiv 2 \pmod{p}$. Thus, one can have $x \equiv 1 \pmod{p}$ or $x^2 - 3 \equiv 1 \pmod{p}$ in the first polynomial, but only $x^3 \equiv 2 \pmod{p}$ in the second polynomial.
\end{sol}
\end{document}